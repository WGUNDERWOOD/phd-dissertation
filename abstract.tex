%! TeX root = dissertation.tex

% TODO 350 words max

% why np is important
Nonparametric methodology is a cornerstone of modern statistics, offering
techniques for analyzing data under minimal assumptions in a broad range of
scenarios. While contemporary procedures such as random forests and kernel
methods are popular due to their performance, flexibility and efficiency, the
statistical properties of such approaches are less well understood in certain
cases. The availability of sound inferential techniques is especially vital,
allowing researchers to not only make high-quality predictions but also to
quantify their uncertainty. We develop and analyze procedures for robust and
practical statistical estimation and inference in some modern nonparametric
settings involving complex estimators and nontraditional data.

% mondrian
We begin in the regression setting where we study the Mondrian random forest, a
recently introduced random forest variant in which the partitions are drawn
from a Mondrian process. We present a comprehensive analysis of the statistical
properties of Mondrian random forests, including a central limit theorem for
the estimated regression function and a characterization of the large-sample
bias. We show how these main results allow for feasible and valid nonparametric
inference such as constructing confidence intervals. We further provide a
debiasing procedure for Mondrian random forests which allow them to achieve
minimax-optimal estimation rates in smooth function classes and with covariates
of arbitrary dimension, assuming appropriate tuning of parameters.

% kernel
Next, we turn our attention to nonparametric estimation with dependent data,
considering dyadic samples associated with the edges of a network. We
investigate the problem of estimating a dyadic Lebesgue density function using
a kernel-based density estimator, and present results for minimax-optimal
estimation and uniform inference. A central feature in dyadic data analysis is
the potential presence of degenerate points, complicating the uniform analysis.
Nonetheless our methods for inference, including uniform confidence bands as
well as counterfactual procedures, remain valid even in such degenerate cases.
We provide illustrations with both simulated and real-world data, and our
technical contributions regarding strong approximations and maximal
inequalities may be of independent interest.

% yurinskii
We devote the final part to developing some new probabilistic results with
applications to nonparametric inference. Coupling theory has become a popular
approach for distributional analysis in recent years, and Yurinskii's coupling
stands out for its wide applicability and explicit formulation.
We present some generalizations of Yurinskii's coupling, treating approximate
martingale data under weaker conditions than previously imposed. We allow for
the coupling variable to follow a Gaussian mixture distribution, and a
third-order method gives faster rates in certain situations. We showcase our
main results with applications to factor models and martingale empirical
processes, as well as nonparametric partitioning-based and local polynomial
regression procedures.
