%! TeX root = dissertation.tex

I am extremely fortunate to have been surrounded by many truly wonderful people
over the course of my career, and without their support this dissertation would
not have been possible. While it is infeasible for me to identify every one of
them individually, I would like to mention a few names in particular, to
recognize those who have been especially important to me during the last few
years.

Firstly, I would like to express my utmost gratitude to my Ph.D.\ adviser,
Matias Cattaneo. Working with Matias has been genuinely inspirational for me,
and I could not have asked for a more rewarding start to my journey as a
researcher. From the very beginning he has guided me expertly through my
studies providing hands-on assistance when required, while also allowing me the
independence necessary to develop as an academic. I hope that, during the four
years we have worked together, I have acquired just a fraction of his formidable
mathematical intuition, keen attention to detail, boundless creativity, and
inimitable pedagogical skill. Alongside his role as my adviser, Matias has been
above all a friend at Princeton, who has been in equal measure inspiring,
insightful, dedicated, understanding, and kind.

Secondly, I would like to thank all of the faculty members at Princeton and
beyond who have acted as my collaborators and mentors, without whom none of my
work would have been possible. In particular, I express my gratitude to my
tireless Ph.D.\ committee members and letter writers Jianqing Fan and Jason
Klusowski, my coauthors Yingjie Feng and Ricardo Masini, my dissertation reader
Boris Hanin, my colleagues Sanjeev Kulkarni and Roc{\'i}o Titiunik, my teachers
Mykhaylo Shkolnikov, Amir Ali Ahmadi and Ramon van Handel, and my former
supervisor Mihai Cucuringu, who started me on my research journey many years
ago. I am grateful also for the staff members at Princeton who have been
perpetually helpful during my Ph.D.\ experience, and would like to identify Kim
Lupinacci in particular; her assistance in all things administrative has been
invaluable.

I am grateful to my fellow graduate students in the ORFE department for their
technical expertise and generosity with their time, and for making Sherrerd
Hall such a vibrant and exciting space; especially Jose Avilez,
Ben Budway, Rajita Chandak,
Abraar Chaudhry, Stefan Clarke, G{\"o}k{\c c}e Dayan{\i}kl{\i}, Nicolas Garcia,
Felix Hoefer, Erica Lai, Jackie Lok, Maya Mutic, Dan Rigobon, Till Saenger,
Rajiv Sambharya, Boris Shigida, Igor Silin, Giang Truong, and Rae Yu. Our
regular social events made a contribution to my well-being which is difficult
to overstate, providing a much-needed respite from research. Thank you also to
all the students who took the various classes I taught, and to my group of
senior thesis undergraduates, for being committed, patient, and responsive, and
for helping me to develop as an instructor.

More broadly, I would like to thank all of my friends at Princeton and beyond
for their unfailing support and reliability through times good and bad, and for
helping to create so many of my treasured memories. In particular,
Ole Agersnap, James Ashford, Christian Baehr,
Chris Bambic, Kevin Beeson, Pier Beneventano, Giulia Crippa, Reece
Edmends, Robin Franklin, Bonnie Ko,
Grace Matthews, Dan Mead, Ben Musachio, Monika
Papayova, Will Pedrick, Oliver Philcox,
Nandita Rao, Alex Rice, Edward Rowe, David Snyder,
Nikitas Tampakis, and Anita Zhang.
Thank you to the Princeton Chapel Choir for being such a wonderful
community of musicians and a source of close friends; I am
grateful particularly for our fantastic directors Nicole Aldrich
and Penna Rose, and organist Eric Plutz.

Lastly but most importantly, I want to thank my family for their unwavering
support throughout my studies. My visits back home have been a source of joy
throughout my long and often challenging Ph.D., and I cherish every moment I
have spent with my parents, sister, grandparents, and extended family.
