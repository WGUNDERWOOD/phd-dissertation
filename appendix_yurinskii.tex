%! TeX root = dissertation.tex

\chapter[Supplement to Yurinskii's Coupling for Martingales]%
{Supplement to Yurinskii's \\ Coupling for Martingales}
\label{app:yurinskii}

\section{Proofs of main results}
\label{sec:yurinskii_proofs}

\subsection{Preliminary lemmas}

We give a sequence of preliminary lemmas which are useful for establishing our
main results. Firstly, we present a conditional version of Strassen's theorem
for the $\ell^p$-norm \citep[Theorem~B.2]{chen2020jackknife}, stated for
completeness as Lemma~\ref{lem:strassen}.

\begin{lemma}[A conditional Strassen theorem for the
  \texorpdfstring{$\ell^p$}{lp}-norm]%
  \label{lem:strassen}
  %
  Let $(\Omega, \cH, \P)$ be a probability space supporting the $\R^d$-valued
  random variable $X$ for some $d \geq 1$. Let $\cH'$ be a countably generated
  sub-$\sigma$-algebra of $\cH$ and suppose there exists a $\Unif[0,1]$ random
  variable on $(\Omega, \cH, \P)$ which is independent of the $\sigma$-algebra
  generated by $X$ and $\cH'$. Consider a regular conditional distribution
  $F(\cdot \mid \cH')$ satisfying the following. Firstly, $F(A \mid \cH')$ is
  an $\cH'$-measurable random variable for all Borel sets $A \in \cB(\R^d)$.
  Secondly, $F(\cdot \mid \cH')(\omega)$ is a Borel probability measure on
  $\R^d$ for all $\omega \in \Omega$. Taking $\eta, \rho > 0$ and
  $p \in [1, \infty]$, with $\E^*$ the outer expectation, if
  %
  \begin{align*}
    \E^* \left[
      \sup_{A \in \cB(\R^d)}
      \Big\{
        \P \big( X \in A \mid \cH' \big)
        - F \big( A_p^\eta \mid \cH' \big)
      \Big\}
    \right]
    \leq \rho,
  \end{align*}
  %
  where $A_p^\eta = \{x \in \R^d : \|x - A\|_p \leq \eta\}$
  and $\|x - A\|_p = \inf_{x' \in A} \|x - x'\|_p$,
  then there exists an $\R^d$-valued random variable $Y$
  with $Y \mid \cH' \sim F(\cdot \mid \cH')$
  and $\P \left( \|X-Y\|_p > \eta \right) \leq \rho$.
  %
\end{lemma}

\begin{proof}[Lemma~\ref{lem:strassen}]
  By Theorem~B.2 in \citet{chen2020jackknife}, noting that the $\sigma$-algebra
  generated by $Z$ is countably generated and using the metric induced by the
  $\ell^p$-norm.
\end{proof}

Next, we present in Lemma~\ref{lem:smooth_approximation} an analytic result
concerning the smooth approximation of Borel set indicator functions, similar
to that given in \citet[Lemma~39]{belloni2019conditional}.

\begin{lemma}[Smooth approximation of Borel indicator functions]%
  \label{lem:smooth_approximation}
  Let $A \subseteq \R^d$ be a Borel set and $Z \sim \cN(0, I_d)$.
  For $\sigma, \eta > 0$ and $p \in [1, \infty]$, define
  %
  \begin{align*}
    g_{A\eta}(x)
    &=
    \left( 1 - \frac{\|x-A^\eta\|_p}{\eta} \right) \vee 0
    &                  &\text{and}
    &f_{A\eta\sigma}(x)
    &=
    \E\big[g_{A\eta}(x + \sigma Z) \big].
  \end{align*}
  %
  Then $f$ is infinitely differentiable
  and with $\varepsilon = \P(\|Z\|_p > \eta / \sigma)$,
  for all $k \geq 0$,
  any multi-index $\kappa = (\kappa_1,\dots, \kappa_d)\in\N^d$,
  and all $x,y \in \R^d$,
  we have $|\partial^\kappa f_{A\eta\sigma}(x)| \leq
  \frac{\sqrt{\kappa!}}{\sigma^{|\kappa|}}$ and
  %
  \begin{align*}
    &\Bigg|
    f_{A\eta\sigma}(x+y) - \sum_{|\kappa| = 0}^k
    \frac{1}{\kappa!}
    \partial^\kappa f_{A\eta\sigma}(x)
    y^\kappa
    \Bigg|
    \leq
    \frac{\|y\|_p \|y\|_2^k}{\sigma^k \eta \sqrt{k!}}, \\
    &(1 - \varepsilon) \I\big\{x \in A\big\}
    \leq f_{A\eta\sigma}(x)
    \leq \varepsilon + (1 - \varepsilon)
    \I\big\{x \in A^{3\eta}\big\}.
  \end{align*}
  %
\end{lemma}

\begin{proof}[Lemma~\ref{lem:smooth_approximation}]
  Drop the subscripts on $g_{A\eta}$ and $f_{A \eta \sigma}$.
  By Taylor's theorem with Lagrange remainder, for a $t \in [0,1]$,
  %
  \begin{align*}
    \Bigg|
    f(x + y)
    - \sum_{|\kappa|=0}^{k}
    \frac{1}{\kappa!}
    \partial^{\kappa} f(x)
    y^\kappa
    \Bigg|
    \leq
    \Bigg|
    \sum_{|\kappa|=k}
    \frac{y^\kappa}{\kappa!}
    \big(
      \partial^{\kappa} f(x + t y)
      - \partial^{\kappa} f(x)
    \big)
    \Bigg|.
  \end{align*}
  %
  Now with $\phi(x) = \frac{1}{\sqrt{2 \pi}} e^{-x^2/2}$,
  %
  \begin{align*}
    f(x)
    &=
    \E\big[g(x + \sigma W) \big]
    =
    \int_{\R^d}
    g(x + \sigma u)
    \prod_{j=1}^{d}
    \phi(u_j)
    \diff u
    =
    \frac{1}{\sigma^d}
    \int_{\R^d}
    g(u)
    \prod_{j=1}^{d}
    \phi \left( \frac{u_j-x_j}{\sigma} \right)
    \diff u
  \end{align*}
  %
  and since the integrand is bounded, we exchange differentiation and
  integration to compute
  %
  \begin{align}
    \nonumber
    \partial^\kappa
    f(x)
    &=
    \frac{1}{\sigma^{d+|\kappa|}}
    \int_{\R^d}
    g(u)
    \prod_{j=1}^{d}
    \partial^{\kappa_j}
    \phi \left( \frac{u_j-x_j}{\sigma} \right)
    \diff u \\
    \nonumber
    &=
    \left( \frac{-1}{\sigma} \right)^{|\kappa|}
    \hspace*{-1mm}
    \int_{\R^d}
    g(x + \sigma u)
    \prod_{j=1}^{d}
    \partial^{\kappa_j}
    \phi(u_j)
    \diff u \\
    \label{eq:smoothing_derivative}
    &=
    \left( \frac{-1}{\sigma} \right)^{|\kappa|}
    \E \Bigg[
      g(x + \sigma Z)
      \prod_{j=1}^{d}
      \frac{\partial^{\kappa_j}\phi(Z_j)}{\phi(Z_j)}
    \Bigg],
  \end{align}
  %
  where $Z \sim \cN(0, I_d)$.
  Recalling that $|g(x)| \leq 1$ and applying the Cauchy--Schwarz inequality,
  %
  \begin{align*}
    \left|
    \partial^\kappa
    f(x)
    \right|
    &\leq
    \frac{1}{\sigma^{|\kappa|}}
    \prod_{j=1}^{d}
    \E \left[
      \left(
        \frac{\partial^{\kappa_j}\phi(Z_j)}{\phi(Z_j)}
      \right)^2
    \right]^{1/2}
    \leq
    \frac{1}{\sigma^{|\kappa|}}
    \prod_{j=1}^{d}
    \sqrt{\kappa_j!}
    =
    \frac{\sqrt{\kappa!}}{\sigma^{|\kappa|}},
  \end{align*}
  %
  as the expected square of the Hermite polynomial of degree
  $\kappa_j$ against the standard Gaussian measure is $\kappa_j!$. By the
  reverse triangle inequality, $|g(x + t y) - g(x)| \leq t \|y\|_p / \eta$,
  so by \eqref{eq:smoothing_derivative},
  %
  \begin{align*}
    &\left|
    \sum_{|\kappa|=k}
    \frac{y^\kappa}{\kappa!}
    \big(
      \partial^{\kappa} f(x + t y)
      - \partial^{\kappa} f(x)
    \big)
    \right| \\
    &\quad=
    \left|
    \sum_{|\kappa|=k}
    \frac{y^\kappa}{\kappa!}
    \frac{1}{\sigma^{|\kappa|}}
    \E \Bigg[
      \big(
        g(x + t y + \sigma Z)
        - g(x + \sigma Z)
      \big)
      \prod_{j=1}^{d}
      \frac{\partial^{\kappa_j}\phi(Z_j)}{\phi(Z_j)}
    \Bigg]
    \right| \\
    &\quad\leq
    \frac{t \|y\|_p}{\sigma^k \eta}
    \, \E \left[
      \Bigg|
      \sum_{|\kappa|=k}
      \frac{y^\kappa}{\kappa!}
      \prod_{j=1}^{d}
      \frac{\partial^{\kappa_j}\phi(Z_j)}{\phi(Z_j)}
      \Bigg|
    \right].
  \end{align*}
  %
  Therefore by the Cauchy--Schwarz inequality,
  %
  \begin{align*}
    &\Bigg(
      \sum_{|\kappa|=k}
      \frac{y^\kappa}{\kappa!}
      \big(
        \partial^{\kappa} f(x + t y)
        - \partial^{\kappa} f(x)
      \big)
    \Bigg)^2 \\
    &\quad\leq
    \frac{t^2 \|y\|_p^2}{\sigma^{2k} \eta^2}
    \, \E \left[
      \Bigg(
        \sum_{|\kappa|=k}
        \frac{y^\kappa}{\kappa!}
        \prod_{j=1}^{d}
        \frac{\partial^{\kappa_j} \phi(Z_j)}{\phi(Z_j)}
      \Bigg)^2
    \right] \\
    &\quad=
    \frac{t^2 \|y\|_p^2}{\sigma^{2k} \eta^2}
    \sum_{|\kappa|=k}
    \sum_{|\kappa'|=k}
    \frac{y^{\kappa + \kappa'}}{\kappa! \kappa'!}
    \prod_{j=1}^{d}
    \, \E \left[
      \frac{\partial^{\kappa_j} \phi(Z_j)}{\phi(Z_j)}
      \frac{\partial^{\kappa'_j} \phi(Z_j)}{\phi(Z_j)}
    \right].
  \end{align*}
  %
  Orthogonality of Hermite polynomials gives zero if
  $\kappa_j \neq \kappa'_j$. By the multinomial theorem,
  %
  \begin{align*}
    \left|
    f(x + y)
    - \sum_{|\kappa|=0}^{k}
    \frac{1}{\kappa!}
    \partial^{\kappa} f(x)
    y^\kappa
    \right|
    &\leq
    \frac{\|y\|_p}{\sigma^k \eta}
    \Bigg(
      \sum_{|\kappa|=k}
      \frac{y^{2 \kappa}}{\kappa!}
    \Bigg)^{1/2}
    \leq
    \frac{\|y\|_p}{\sigma^k \eta \sqrt{k!}}
    \Bigg(
      \sum_{|\kappa|=k}
      \frac{k!}{\kappa!}
      y^{2 \kappa}
    \Bigg)^{1/2} \\
    &\leq
    \frac{\|y\|_p \|y\|_2^k}{\sigma^k \eta \sqrt{k!}}.
  \end{align*}
  %
  For the final result, since
  $f(x) = \E \left[ g(x + \sigma Z) \right]$ and
  $\I\big\{x \in A^\eta\big\}\leq g(x)\leq \I\big\{x \in A^{2\eta}\big\}$,
  %
  \begin{align*}
    f(x)
    &\leq
    \P \left( x + \sigma Z \in A^{2 \eta} \right) \\
    &\leq
    \P \left( \|Z\|_p > \frac{\eta}{\sigma} \right)
    + \I \left\{ x \in A^{3 \eta} \right\}
    \P \left( \|Z\|_p \leq \frac{\eta}{\sigma} \right)
    = \varepsilon
    + (1 - \varepsilon) \I \left\{ x \in A^{3 \eta} \right\}
    \hspace*{-1mm}, \\
    f(x)
    &\geq
    \P \left( x + \sigma Z \in A^{\eta} \right)
    \leq
    \I \left\{ x \in A \right\}
    \P \left( \|Z\|_p \leq \frac{\eta}{\sigma} \right)
    = (1 - \varepsilon) \I \left\{ x \in A \right\}.
  \end{align*}
  %
\end{proof}

We provide a useful Gaussian inequality in Lemma~\ref{lem:gaussian_useful}
which helps bound the $\beta_{\infty,k}$ moment terms appearing in several
places throughout the analysis.

\begin{lemma}[A useful Gaussian inequality]%
  \label{lem:gaussian_useful}

  Let $X \sim \cN(0, \Sigma)$
  where $\sigma_j^2 = \Sigma_{j j} \leq \sigma^2$ for all $1 \leq j \leq d$.
  Then
  %
  \begin{align*}
    \E\left[
      \|X\|_2^2
      \|X\|_\infty
    \right]
    &\leq
    4 \sigma \sqrt{\log 2d}
    \,\sum_{j=1}^d \sigma_j^2
    &&\text{and}
    &\E\left[
      \|X\|_2^3
      \|X\|_\infty
    \right]
    &\leq
    8 \sigma \sqrt{\log 2d}
    \,\bigg( \sum_{j=1}^d \sigma_j^2 \bigg)^{3/2}.
  \end{align*}
  %
\end{lemma}

\begin{proof}[Lemma~\ref{lem:gaussian_useful}]

  By Cauchy--Schwarz, with $k \in \{2,3\}$, we have
  $\E\left[\|X\|_2^{k} \|X\|_\infty \right]
  \leq \E\big[\|X\|_2^{2k} \big]^{1/2} \E\big[\|X\|_\infty^2 \big]^{1/2}$.
  For the first term, by H{\"o}lder's inequality and the fourth and sixth
  moments of the normal distribution,
  %
  \begin{align*}
    \E\big[\|X\|_2^4 \big]
    &=
    \E\Bigg[
      \bigg(
        \sum_{j=1}^d X_j^2
      \bigg)^2
    \Bigg]
    =
    \sum_{j=1}^d \sum_{k=1}^d
    \E\big[
      X_j^2 X_k^2
    \big]
    \leq
    \bigg(
      \sum_{j=1}^d
      \E\big[X_j^4 \big]^{\frac{1}{2}}
    \bigg)^2
    =
    3 \bigg(
      \sum_{j=1}^d
      \sigma_j^2
    \bigg)^2, \\
    \E\big[\|X\|_2^6 \big]
    &=
    %\E\Bigg[
    %\bigg(
    %\sum_{j=1}^d X_j^2
    %\bigg)^3
    %\Bigg]
    %=
    \sum_{j=1}^d \sum_{k=1}^d \sum_{l=1}^d
    \E\big[
      X_j^2 X_k^2 X_l^2
    \big]
    \leq
    \bigg(
      \sum_{j=1}^d
      \E\big[X_j^6 \big]^{\frac{1}{3}}
    \bigg)^3
    =
    15 \bigg(
      \sum_{j=1}^d
      \sigma_j^2
    \bigg)^3.
  \end{align*}
  %
  For the second term, by Jensen's inequality and the $\chi^2$ moment
  generating function,
  %
  \begin{align*}
    \E\big[\|X\|_\infty^2 \big]
    &=
    \E\left[
      \max_{1 \leq j \leq d}
      X_j^2
    \right]
    \leq
    4 \sigma^2
    \log
    \sum_{j=1}^d
    \E\Big[
      e^{X_j^2 / (4\sigma^2)}
    \Big]
    \leq
    4 \sigma^2
    \log
    \sum_{j=1}^d
    \sqrt{2}
    \leq
    4 \sigma^2
    \log 2 d.
  \end{align*}
  %
\end{proof}

We provide an $\ell^p$-norm tail probability bound for Gaussian variables in
Lemma~\ref{lem:gaussian_pnorm}, motivating the definition of the term
$\phi_p(d)$.

\begin{lemma}[Gaussian \texorpdfstring{$\ell^p$}{lp}-norm bound]%
  \label{lem:gaussian_pnorm}
  Let $X \sim \cN(0, \Sigma)$ where $\Sigma \in \R^{d \times d}$
  is a positive semi-definite matrix. Then
  $\E\left[ \|X\|_p \right]
  \leq
  \phi_p(d)
  \max_{1 \leq j \leq d}
  \sqrt{\Sigma_{j j}}$
  where $\phi_p(d) = \sqrt{pd^{2/p} }$ for $p \in [1,\infty)$
  and $\phi_\infty(d) = \sqrt{2\log 2d}$.
\end{lemma}

\begin{proof}[Lemma~\ref{lem:gaussian_pnorm}]

  For $p \in [1, \infty)$,
  as each $X_j$ is Gaussian, we have
  $\big(\E\big[|X_j|^p\big]\big)^{1/p}
  \leq \sqrt{p\, \E[X_j^2]}
  = \sqrt{p \Sigma_{j j}}$.
  Therefore
  %
  \begin{align*}
    \E\big[\|X\|_p\big]
    &\leq
    \Bigg(\sum_{j=1}^d \E \big[ |X_j|^p \big] \Bigg)^{1/p}
    \leq \Bigg(\sum_{j=1}^d p^{p/2} \Sigma_{j j}^{p/2} \Bigg)^{1/p}
    \leq \sqrt{p d^{2/p}}
    \max_{1\leq j\leq d}
    \sqrt{\Sigma_{j j}}
  \end{align*}
  %
  by Jensen's inequality.
  For $p=\infty$,
  with $\sigma^2 = \max_j \Sigma_{j j}$,
  for $t>0$,
  %
  \begin{align*}
    \E\big[\|X\|_\infty \big]
    &\leq
    t
    \log
    \sum_{j=1}^d
    \E\Big[
      e^{|X_j| / t}
    \Big]
    \leq
    t
    \log
    \sum_{j=1}^d
    \E\Big[
      2 e^{X_j / t}
    \Big]
    \leq t \log \Big(2 d e^{\sigma^2/(2t^2)}\Big) \\
    &\leq t \log 2 d + \frac{\sigma^2}{2t},
  \end{align*}
  %
  again by Jensen's inequality.
  Setting $t = \frac{\sigma}{\sqrt{2 \log 2d}}$ gives
  $\E\big[\|X\|_\infty \big] \leq \sigma \sqrt{2 \log 2d}$.
  %
\end{proof}

We give a Gaussian--Gaussian $\ell^p$-norm approximation
as Lemma~\ref{lem:feasible_gaussian}, useful for
ensuring approximations remain valid upon substituting
an estimator for the true variance matrix.

\begin{lemma}[Gaussian--Gaussian approximation in
  \texorpdfstring{$\ell^p$}{lp}-norm]%
  \label{lem:feasible_gaussian}

  Let $\Sigma_1, \Sigma_2 \in \R^{d \times d}$ be positive semi-definite
  and take $Z \sim \cN(0, I_d)$.
  For $p \in [1, \infty]$ we have
  %
  \begin{align*}
    \P\left(
      \left\|
      \left(\Sigma_1^{1/2} - \Sigma_2^{1/2}\right) Z
      \right\|_p
      > t
    \right)
    &\leq
    2 d \exp \left(
      \frac{-t^2}
      {2 d^{2/p} \big\|\Sigma_1^{1/2} - \Sigma_2^{1/2}\big\|_2^2}
    \right).
  \end{align*}

\end{lemma}

\begin{proof}[Lemma~\ref{lem:feasible_gaussian}]

  Let $\Sigma \in \R^{d \times d}$ be positive semi-definite
  and write $\sigma^2_j = \Sigma_{j j} $.
  For $p \in [1, \infty)$ by a union bound and
  Gaussian tail probabilities,
  %
  \begin{align*}
    \P\left(\big\| \Sigma^{1/2} Z \big\|_p > t \right)
    &=
    \P\Bigg(
      \sum_{j=1}^d
      \left|
      \left(
        \Sigma^{1/2} Z
      \right)_j
      \right|^p
    > t^p \Bigg)
    \leq
    \sum_{j=1}^d
    \P\Bigg(
      \left|
      \left(
        \Sigma^{1/2} Z
      \right)_j
      \right|^p
      > \frac{t^p \sigma_j^p}{\|\sigma\|_p^p}
    \Bigg) \\
    &=
    \sum_{j=1}^d
    \P\Bigg(
      \left|
      \sigma_j Z_j
      \right|^p
      > \frac{t^p \sigma_j^p}{\|\sigma\|_p^p}
    \Bigg)
    =
    \sum_{j=1}^d
    \P\left(
      \left| Z_j \right|
      > \frac{t}{\|\sigma\|_p}
    \right) \\
    &\leq
    2 d \,
    \exp\left( \frac{-t^2}{2 \|\sigma\|_p^2} \right).
  \end{align*}
  %
  The same result holds for $p = \infty$ since
  %
  \begin{align*}
    \P\left(\big\| \Sigma^{1/2} Z \big\|_\infty > t \right)
    &=
    \P\left(
      \max_{1 \leq j \leq d}
      \left|
      \left(
        \Sigma^{1/2} Z
      \right)_j
      \right|
    > t \right)
    \leq
    \sum_{j=1}^d
    \P\left(
      \left|
      \left(
        \Sigma^{1/2} Z
      \right)_j
      \right|
      > t
    \right) \\
    &=
    \sum_{j=1}^d
    \P\left(
      \left|
      \sigma_j Z_j
      \right|
      > t
    \right)
    \leq
    2 \sum_{j=1}^d
    \exp\left( \frac{-t^2}{2 \sigma_j^2} \right)
    \leq
    2 d
    \exp\left( \frac{-t^2}{2 \|\sigma\|_\infty^2} \right).
  \end{align*}
  %
  Now we apply this to the matrix
  $\Sigma = \big(\Sigma_1^{1/2} - \Sigma_2^{1/2}\big)^2$.
  For $p \in [1, \infty)$,
  %
  \begin{align*}
    \|\sigma\|_p^p
    &=
    \sum_{j=1}^d (\Sigma_{j j})^{p/2}
    =
    \sum_{j=1}^d
    \Big(\big(\Sigma_1^{1/2} - \Sigma_2^{1/2}\big)^2\Big)_{j j}^{p/2}
    \leq
    d \max_{1 \leq j \leq d}
    \Big(\big(\Sigma_1^{1/2} - \Sigma_2^{1/2}\big)^2\Big)_{j j}^{p/2} \\
    &\leq
    d \, \Big\|\big(\Sigma_1^{1/2} - \Sigma_2^{1/2}\big)^2\Big\|_2^{p/2}
    =
    d \, \big\|\Sigma_1^{1/2} - \Sigma_2^{1/2}\big\|_2^p
  \end{align*}
  %
  Similarly for $p = \infty$ we have
  %
  \begin{align*}
    \|\sigma\|_\infty
    &=
    \max_{1 \leq j \leq d}
    (\Sigma_{j j})^{1/2}
    =
    \max_{1 \leq j \leq d}
    \Big(\big(\Sigma_1^{1/2} - \Sigma_2^{1/2}\big)^2\Big)_{j j}^{1/2}
    \leq
    \big\|\Sigma_1^{1/2} - \Sigma_2^{1/2}\big\|_2.
  \end{align*}
  %
  Thus for all $p \in [1, \infty]$ we have
  $\|\sigma\|_p \leq
  d^{1/p} \big\|\Sigma_1^{1/2} - \Sigma_2^{1/2}\big\|_2$,
  with $d^{1/\infty} = 1$. Hence
  %
  \begin{align*}
    \P\left(
      \left\|
      \left(\Sigma_1^{1/2} - \Sigma_2^{1/2}\right) Z
      \right\|_p
      > t
    \right)
    &\leq
    2 d \exp \left( \frac{-t^2}{2 \|\sigma\|_p^2} \right)
    \leq
    2 d \exp \left(
      \frac{-t^2}
      {2 d^{2/p} \big\|\Sigma_1^{1/2} - \Sigma_2^{1/2}\big\|_2^2}
    \right).
  \end{align*}
  %
\end{proof}

We also include, for completeness, a variance bound
(Lemma~\ref{lem:variance_mixing})
and an exponential concentration inequality
(Lemma~\ref{lem:exponential_mixing})
for $\alpha$-mixing random variables.

\begin{lemma}[Variance bounds for
  \texorpdfstring{$\alpha$}{alpha}-mixing random variables]
  \label{lem:variance_mixing}

  Let $X_1, \ldots, X_n$ be
  real-valued $\alpha$-mixing random
  variables with mixing coefficients $\alpha(j)$.
  Then
  %
  \begin{enumerate}[label=(\roman*)]

    \item
      \label{eq:variance_mixing_bounded}
      If for constants $M_i$ we have
      $|X_i| \leq M_i$ a.s.\ then
      %
      \begin{align*}
        \Var\left[
          \sum_{i=1}^n X_i
        \right]
        &\leq
        4 \sum_{j=1}^\infty \alpha(j)
        \sum_{i=1}^n M_i^2.
      \end{align*}

    \item
      \label{eq:variance_mixing_exponential}
      If $\alpha(j) \leq e^{-2j / C_\alpha}$ then
      for any $r>2$ there is a constant
      $C_r$ depending only on $r$
      such that
      %
      \begin{align*}
        \Var\left[
          \sum_{i=1}^n X_i
        \right]
        &\leq
        C_r C_\alpha
        \sum_{i=1}^n
        \E\big[|X_i|^r\big]^{2/r}.
      \end{align*}
  \end{enumerate}
  %
\end{lemma}

\begin{proof}[Lemma~\ref{lem:variance_mixing}]

  Define
  $\alpha^{-1}(t) =
  \inf\{j \in \N : \alpha(j) \leq t\}$
  and
  $Q_i(t) = \inf\{s \in \R : \P(|X_i| > s) \leq t\}$.
  By Corollary~1.1 in \citet{rio2017asymptotic}
  and H{\"o}lder's inequality for $r > 2$,
  %
  \begin{align*}
    \Var\left[
      \sum_{i=1}^n X_i
    \right]
    &\leq
    4 \sum_{i=1}^n
    \int_0^1 \alpha^{-1}(t)
    Q_i(t)^2 \diff{t} \\
    &\leq
    4 \sum_{i=1}^n
    \left(
      \int_0^1 \alpha^{-1}(t)^{\frac{r}{r-2}} \diff{t}
    \right)^{\frac{r-2}{r}}
    \left(
      \int_0^1 |Q_i(t)|^r \diff{t}
    \right)^{\frac{2}{r}}
    \diff{t}.
  \end{align*}
  %
  Now note that if $U \sim \Unif[0,1]$ then
  $Q_i(U)$ has the same distribution as $X_i$.
  Therefore
  %
  \begin{align*}
    \Var\left[
      \sum_{i=1}^n X_i
    \right]
    &\leq
    4
    \left(
      \int_0^1 \alpha^{-1}(t)^{\frac r{r-2}} \diff{t}
    \right)^{\frac{r-2}r}
    \sum_{i=1}^n
    \E[|X_i|^r]^{\frac 2 r}.
  \end{align*}
  %
  If $\alpha(j) \leq e^{-2j/C_\alpha}$ then
  $\alpha^{-1}(t) \leq \frac{-C_\alpha \log t}{2}$
  so, for some constant
  $C_r$ depending only on $r$,
  %
  \begin{align*}
    \Var\left[
      \sum_{i=1}^n X_i
    \right]
    \leq
    2 C_\alpha
    \left(
      \int_0^1 (-\log t)^{\frac r{r-2}} \diff{t}
    \right)^{\frac{r-2} r}
    \sum_{i=1}^n
    \E[|X_i|^r]^{\frac 2 r}
    \leq
    C_r C_\alpha
    \sum_{i=1}^n
    \E[|X_i|^r]^{\frac 2 r}.
  \end{align*}
  %
  Alternatively, if for constants $M_i$ we have
  $|X_i| \leq M_i$ a.s.\ then
  %
  \begin{align*}
    \Var\left[
      \sum_{i=1}^n X_i
    \right]
    &\leq
    4 \int_0^1 \alpha^{-1}(t)
    \diff{t}
    \sum_{i=1}^n M_i^2
    \leq
    4 \sum_{j=1}^\infty \alpha(j)
    \sum_{i=1}^n M_i^2.
  \end{align*}
  %
\end{proof}

\begin{lemma}[Exponential concentration inequalities for
  \texorpdfstring{$\alpha$}{alpha}-mixing random variables]
  \label{lem:exponential_mixing}

  Let $X_1, \ldots, X_n$ be zero-mean real-valued
  variables with $\alpha$-mixing coefficients
  $\alpha(j) \leq e^{-2 j / C_\alpha}$.

  \begin{enumerate}[label=(\roman*)]

    \item
      \label{eq:exponential_mixing_bounded}
      Suppose $|X_i| \leq M$ a.s.\ for each $1 \leq i \leq n$.
      Then for all $t > 0$ there is a constant $C_1$ with
      %
      \begin{align*}
        \P\left(
          \left|
          \sum_{i=1}^n
          X_i
          \right|
          > C_1 M \big( \sqrt{n t}
          + (\log n)(\log \log n) t \big)
        \right)
        &\leq
        C_1 e^{-t}.
      \end{align*}
      %
    \item
      \label{eq:exponential_mixing_bernstein}
      Suppose further
      $\sum_{j=1}^n |\Cov[X_i, X_j]| \leq \sigma^2$.
      Then for all $t > 0$ there is a constant $C_2$ with
      %
      \begin{align*}
        \P\left(
          \left|
          \sum_{i=1}^n
          X_i
          \right|
          \geq C_2 \big( (\sigma \sqrt n + M) \sqrt t
          + M (\log n)^2 t \big)
        \right)
        &\leq
        C_2 e^{-t}.
      \end{align*}

  \end{enumerate}

\end{lemma}

\begin{proof}[Lemma~\ref{lem:exponential_mixing}]

  We apply results from \citet{merlevede2009bernstein},adjusting constants
  where necessary.
  %
  \begin{enumerate}[label=(\roman*)]

    \item
      By Theorem~1 in \citet{merlevede2009bernstein},
      %
      \begin{align*}
        \P\left(
          \left|
          \sum_{i=1}^n
          X_i
          \right|
          > t
        \right)
        &\leq
        \exp\left(
          -\frac{C_1 t^2}{n M^2 + Mt (\log n)(\log\log n)}
        \right).
      \end{align*}
      %
      Replace $t$ by
      $M \sqrt{n t} + M (\log n)(\log \log n) t$.

    \item
      By Theorem~2 in \citet{merlevede2009bernstein},
      %
      \begin{align*}
        \P\left(
          \left|
          \sum_{i=1}^n
          X_i
          \right|
          > t
        \right)
        &\leq
        \exp\left(
          -\frac{C_2 t^2}{n\sigma^2 + M^2 + Mt (\log n)^2}
        \right).
      \end{align*}
      %
      Replace $t$ by
      $\sigma \sqrt n \sqrt t + M \sqrt t + M (\log n)^2 t$.
  \end{enumerate}
  %
\end{proof}

\subsection{Main results}

To establish Theorem~\ref{thm:sa_dependent}, we first
give the analogous result
for martingales as Lemma~\ref{lem:sa_martingale}. Our approach is similar to
that used in modern versions of Yurinskii's coupling for independent data, as
in Theorem~1 in \citet{lecam1988} and Theorem~10 in Chapter~10 of
\citet{pollard2002user}. The proof of Lemma~\ref{lem:sa_martingale} relies on
constructing a ``modified'' martingale, which is close to the original
martingale, but which has an $\cH_0$-measurable terminal quadratic variation.

\begin{lemma}[Strong approximation for vector-valued martingales]%
  \label{lem:sa_martingale}

  Let $X_1, \ldots, X_n$ be $\R^d$-valued
  square-integrable random vectors
  adapted to a countably generated
  filtration $\cH_0, \ldots, \cH_n$.
  Suppose that
  $\E[X_i \mid \cH_{i-1}] = 0$ for all $1 \leq i \leq n$
  and define the martingale $S = \sum_{i=1}^n X_i$.
  Let $V_i = \Var[X_i \mid \cH_{i-1}]$ and
  $\Omega = \sum_{i=1}^n V_i - \Sigma$
  where $\Sigma$ is a positive semi-definite
  $\cH_0$-measurable $d \times d$ random matrix.
  For each $\eta > 0$ and $p \in [1,\infty]$
  there is $T \mid \cH_0 \sim \cN(0, \Sigma)$ with
  %
  \begin{align*}
    \P\big(\|S-T\|_p > 5\eta\big)
    &\leq
    \inf_{t>0}
    \left\{
      2 \P\big( \|Z\|_p > t \big)
      + \min\left\{
        \frac{\beta_{p,2} t^2}{\eta^3},
        \frac{\beta_{p,3} t^3}{\eta^4}
        + \frac{\pi_3 t^3}{\eta^3}
      \right\}
    \right\} \\
    \nonumber
    &\quad+
    \inf_{M \succeq 0}
    \big\{ 2\gamma(M) + \delta_p(M,\eta)
    + \varepsilon_p(M, \eta)\big\},
  \end{align*}
  %
  where the second infimum is over all positive semi-definite
  $d \times d$ non-random matrices, and
  %
  \begin{align*}
    \beta_{p,k}
    &=
    \sum_{i=1}^n \E\left[\| X_i \|^k_2 \| X_i \|_p
    + \|V_i^{1/2} Z_i \|^k_2 \|V_i^{1/2} Z_i \|_p \right],
    \qquad\gamma(M)
    = \P\big(\Omega \npreceq M\big), \\
    \delta_p(M,\eta)
    &=
    \P\left(
      \big\|\big((\Sigma +M)^{1/2}- \Sigma^{1/2}\big) Z\big\|_p
      \geq \eta
    \right),
    \qquad\pi_3
    =
    \sum_{i=1}^{n+m}
    \sum_{|\kappa| = 3}
    \E \Big[ \big|
      \E \left[ X_i^\kappa \mid \cH_{i-1} \right]
    \big| \Big], \\
    \varepsilon_p(M, \eta)
    &=
    \P\left(\big\| (M - \Omega)^{1/2} Z \big\|_p\geq \eta, \
    \Omega \preceq M\right),
  \end{align*}
  %
  for $k \in \{2,3\}$, with $Z, Z_1,\dots ,Z_n$ i.i.d.\ standard Gaussian
  on $\R^d$ independent of $\cH_n$.
\end{lemma}

\begin{proof}[Lemma~\ref{lem:sa_martingale}]

  \proofparagraph{constructing a modified martingale}

  Take $M \succeq 0$ a fixed positive semi-definite
  $d \times d$ matrix.
  We start by constructing a new martingale based on $S$
  whose quadratic variation is $\Sigma + M$.
  Take $m \geq 1$ and define
  %
  \begin{align*}
    H_k
    &=
    \Sigma
    + M
    - \sum_{i=1}^{k} V_i,
    \qquad\qquad\qquad\qquad\tau
    =
    \sup \big\{ k\in\{0,1,\dots,n\} : H_k \succeq 0 \big\}, \\
    \tilde X_i
    &=
    X_i\I\{i \leq \tau\}
    + \frac{1}{\sqrt{m}} H_\tau^{1/2} Z_i\I\{n+1 \leq i \leq n+m\},
    \qquad\qquad\tilde S
    =
    \sum_{i=1}^{n+m} \tilde X_i,
  \end{align*}
  %
  where $Z_{n+1}, \ldots, Z_{n+m}$ is an i.i.d.\
  sequence of standard Gaussian vectors in $\R^d$
  independent of $\cH_n$,
  noting that $H_0 = \Sigma + M \succeq 0$ a.s.
  Define the filtration
  $\tilde \cH_0, \ldots, \tilde \cH_{n+m}$,
  where $\tilde \cH_i = \cH_i$ for $0 \leq i \leq n$
  and is the $\sigma$-algebra generated by
  $\cH_n$ and $Z_{n+1}, \dots, Z_{i}$ for $n+1 \leq i\leq n+m$.
  Observe that $\tau$ is a stopping time with respect to $\tilde\cH_i$
  because $H_{i+1} - H_i = -V_{i+1} \preceq 0$ almost surely,
  so $\{\tau \leq i\} = \{H_{i+1} \nsucceq 0\}$ for $0\leq i<n$.
  This depends only on $V_1, \dots, V_{i+1}$ and $\Sigma$
  which are $\tilde\cH_i$-measurable.
  Similarly, $\{\tau = n\} = \{H_n \succeq 0\} \in \tilde\cH_{n-1}$.
  Let $\tilde V_i = V_i \I\{i\leq\tau\}$ for
  $1\leq i\leq n$ and
  $\tilde V_i = H_\tau/m$ for $n+1\leq i\leq n+m$.
  Note that $\tilde X_i$ is $\tilde \cH_i$-measurable
  and $\tilde V_i$ is $\tilde \cH_{i-1}$-measurable.
  Further, $\E \left[ \tilde X_i \mid \tilde \cH_{i-1} \right] = 0$ and
  $\E \left[ \tilde X_i \tilde X_i^\T \mid \tilde \cH_{i-1} \right]
  = \tilde V_i$.

  \proofparagraph{bounding the difference between the original and
  modified martingales}

  By the triangle inequality,
  %
  \begin{align*}
    \|S - \tilde S \|_p
    &\leq
    \left\| \sum_{i=\tau+1}^n  X_i \right\|_p
    + \left\| \frac{1}{\sqrt{m}} \sum_{i=n+1}^m H_\tau^{1/2} Z_i \right\|_p.
  \end{align*}
  The first term on the right vanishes on
  $\{\tau = n\} = \{H_n \succeq 0\} = \{\Omega \preceq M\}$.
  For the second term, note that
  $\tfrac{1}{\sqrt{m}}\sum_{i=n+1}^m H_\tau^{1/2} Z_i$
  is distributed as $H_\tau^{1/2}Z$,
  where $Z$ is an independent standard Gaussian variable.
  Also
  $\P\big( \| H_\tau^{1/2} Z \|_p > \eta \big)
  \leq \P\big( \| H_n^{1/2} Z \|_p > \eta,\, \Omega \preceq M)
  + \P\big( \Omega \npreceq M \big)$.
  Therefore
  %
  \begin{equation}%
    \label{eq:approx_modified_original}
    \P\big( \| S - \tilde S \|_p > \eta\big)
    \leq
    2 \P\big(\Omega \npreceq M \big)
    + \P\big( \| (M-\Omega)^{1/2}Z \|_p > \eta,\,
    \Omega \preceq M \big)
    = 2 \gamma(M) + \varepsilon_p(M, \eta).
  \end{equation}

  \proofparagraph{strong approximation of the modified martingale}

  Let $\tilde Z_1, \ldots, \tilde Z_{n+m}$ be i.i.d.\ $\cN(0, I_d)$
  and independent of $\tilde \cH_{n+m}$.
  Define $\check X_i = \tilde V_i^{1/2} \tilde Z_i$
  and $\check S = \sum_{i=1}^{n+m} \check X_i$.
  Fix a Borel set $A \subseteq \R^d$ and $\sigma, \eta > 0$ and
  let $f = f_{A\eta\sigma}$ be the function defined in
  Lemma~\ref{lem:smooth_approximation}.
  By the Lindeberg method, write the telescoping sum
  %
  \begin{align*}
    \E\Big[f\big(\tilde S\big) - f\big(\check S\big)
    \mid \cH_0 \Big]
    &=
    \sum_{i=1}^{n+m}
    \E\Big[ f\big(Y_i + \tilde X_i\big)
      - f\big(Y_i + \check X_i\big)
    \mid \cH_0 \Big]
  \end{align*}
  %
  where
  $Y_i = \sum_{j=1}^{i-1} \tilde X_j + \sum_{j=i+1}^{n+m} \check X_j$.
  % Will: I fixed this typo
  By Lemma~\ref{lem:smooth_approximation} we have for $k \geq 0$
  %
  \begin{align*}
    &\Bigg|
    \E\big[
      f(Y_i + \tilde X_i)
      - f(Y_i + \check X_i)
      \mid \cH_0
    \big]
    - \sum_{|\kappa| = 0}^k
    \frac{1}{\kappa!}
    \E \left[
      \partial^\kappa f(Y_i)
      \left( \tilde X_i^\kappa - \check X_i^\kappa \right)
      \bigm| \cH_0
    \right]
    \Bigg| \\
    &\quad\leq
    \frac{1}{\sigma^k \eta \sqrt{k!}}
    \E \left[
      \|\tilde X_i\|_p \|\tilde X_i\|_2^k
      + \|\check X_i\|_p \|\check X_i\|_2^k
      \bigm| \cH_0
    \right].
  \end{align*}
  %
  With $k \in \{2, 3\}$, we bound each summand.
  With $|\kappa| = 0$ we have
  $\tilde X_i^\kappa = \check X_i^\kappa$,
  so consider $|\kappa| = 1$.
  Noting that $\sum_{i=1}^{n+m} \tilde V_i = \Sigma + M$, define
  %
  \begin{align*}
    \tilde Y_i
    &=
    \sum_{j=1}^{i-1} \tilde X_j
    + \tilde Z_i
    \Bigg(\sum_{j=i+1}^{n+m} \tilde V_j\Bigg)^{1/2}
    =
    \sum_{j=1}^{i-1} \tilde X_j
    + \tilde Z_i
    \Bigg(\Sigma + M - \sum_{j=1}^{i} \tilde V_j\Bigg)^{1/2}
  \end{align*}
  %
  and let $\check \cH_i$ be the $\sigma$-algebra generated by
  $\tilde \cH_{i-1}$ and $\tilde Z_i$.
  Note that $\tilde Y_i$ is $\check \cH_i$-measurable
  and that $Y_i$ and $\tilde Y_i$
  have the same distribution conditional on $\tilde \cH_{n+m}$.
  So
  %
  \begin{align*}
    \sum_{|\kappa| = 1}
    \frac{1}{\kappa!}
    \E\hspace*{-0.5mm}\left[
      \partial^\kappa f(Y_i)
      \big( \tilde X_i^\kappa - \check X_i^\kappa \big)
      \bigm| \cH_0
    \right]
    &=
    \E \left[
      \nabla f(Y_i)^\T
      \big( \tilde X_i - \tilde V_i^{1/2} \tilde Z_i \big)
      \bigm| \cH_0
    \right] \\[-3mm]
    &=
    \E \left[
      \nabla f(\tilde Y_i)^\T \tilde X_i
      \bigm| \cH_0
    \right]
    - \E \left[
      \nabla f(Y_i)^\T \tilde V_i^{1/2} \tilde Z_i
      \bigm| \cH_0
    \right] \\
    &=
    \E \left[
      \nabla f(\tilde Y_i)^\T
      \E \left[
        \tilde X_i
        \mid \check \cH_i
      \right]
      \bigm| \cH_0
    \right] \\
    &\quad-
    \E \left[
      \tilde Z_i
    \right]
    \E \left[
      \nabla f(Y_i)^\T \tilde V_i^{1/2}
      \bigm| \cH_0
    \right] \\
    &=
    \E \left[
      \nabla f(\tilde Y_i)^\T
      \E \left[
        \tilde X_i
        \mid \tilde \cH_{i-1}
      \right]
      \bigm| \cH_0
    \right]
    - 0
    = 0.
  \end{align*}
  %
  Next, if $|\kappa| = 2$ then
  %
  \begin{align*}
    &\sum_{|\kappa| = 2}
    \frac{1}{\kappa!}
    \E \left[
      \partial^\kappa f(Y_i)
      \left( \tilde X_i^\kappa - \check X_i^\kappa \right)
      \bigm| \cH_0
    \right] \\
    &\quad=
    \frac{1}{2}
    \E \left[
      \tilde X_i^\T \nabla^2 f(Y_i) \tilde X_i
      - \tilde Z_i^\T \tilde V_i^{1/2} \nabla^2 f(Y_i)
      \tilde V_i^{1/2} \tilde Z_i
      \bigm| \cH_0
    \right] \\
    &\quad=
    \frac{1}{2}
    \E \left[
      \E \left[
        \Tr \nabla^2 f(\tilde Y_i) \tilde X_i \tilde X_i^\T
        \bigm| \check \cH_i
      \right]
      \bigm| \cH_0
    \right]
    - \frac{1}{2}
    \E \left[
      \Tr \tilde V_i^{1/2} \nabla^2 f(Y_i) \tilde V_i^{1/2}
      \bigm| \cH_0
    \right]
    \E \left[
      \tilde Z_i \tilde Z_i^\T
    \right] \\
    &\quad=
    \frac{1}{2}
    \E \left[
      \Tr \nabla^2 f(Y_i)
      \E \left[
        \tilde X_i \tilde X_i^\T
        \bigm| \tilde \cH_{i-1}
      \right]
      \bigm| \cH_0
    \right]
    - \frac{1}{2}
    \E \left[
      \Tr \nabla^2 f(Y_i) \tilde V_i
      \bigm| \cH_0
    \right]
    = 0.
  \end{align*}
  %
  Finally if $|\kappa| = 3$, then since
  $\check X_i \sim \cN(0, \tilde V_i)$
  conditional on $\tilde \cH_{n+m}$, we have by symmetry of the Gaussian
  distribution and Lemma~\ref{lem:smooth_approximation},
  %
  \begin{align*}
    &
    \left|
    \sum_{|\kappa| = 3}
    \frac{1}{\kappa!}
    \E \left[
      \partial^\kappa f(Y_i)
      \left( \tilde X_i^\kappa - \check X_i^\kappa \right)
      \bigm| \cH_0
    \right]
    \right|
    \\
    &\quad=
    \left|
    \sum_{|\kappa| = 3}
    \frac{1}{\kappa!}
    \left(
      \E \left[
        \partial^\kappa f(\tilde Y_i)
        \E \left[ \tilde X_i^\kappa \mid \check \cH_i \right]
        \bigm| \cH_0
      \right]
      - \E \left[
        \partial^\kappa f(Y_i) \,
        \E \left[
          \check X_i^\kappa
          \bigm| \tilde \cH_{n+m}
        \right]
        \bigm| \cH_0
      \right]
    \right)
    \right|
    \\
    &\quad=
    \left|
    \sum_{|\kappa| = 3}
    \frac{1}{\kappa!}
    \E \left[
      \partial^\kappa f(Y_i) \,
      \E \left[ \tilde X_i^\kappa \mid \tilde \cH_{i-1} \right]
      \bigm| \cH_0
    \right]
    \right|
    \leq
    \frac{1}{\sigma^3}
    \sum_{|\kappa| = 3}
    \E \left[
      \left|
      \E \left[ \tilde X_i^\kappa \mid \tilde \cH_{i-1} \right]
      \right|
      \bigm| \cH_0
    \right].
  \end{align*}
  %
  Combining these and summing over $i$ with $k=2$ shows
  %
  \begin{align*}
    \E\left[
      f\big(\tilde S\big) - f\big(\check S\big)
      \bigm| \cH_0
    \right]
    &\leq
    \frac{1}{\sigma^2 \eta \sqrt{2}}
    \sum_{i=1}^{n+m}
    \E \left[
      \|\tilde X_i\|_p \|\tilde X_i\|_2^2
      + \|\check X_i\|_p \|\check X_i\|_2^2
      \bigm| \cH_0
    \right]
  \end{align*}
  %
  On the other hand, taking $k = 3$ gives
  %
  \begin{align*}
    \E\left[
      f\big(\tilde S\big) - f\big(\check S\big)
      \bigm| \cH_0
    \right]
    &\leq
    \frac{1}{\sigma^3 \eta \sqrt{6}}
    \sum_{i=1}^{n+m}
    \E \left[
      \|\tilde X_i\|_p \|\tilde X_i\|_2^3
      + \|\check X_i\|_p \|\check X_i\|_2^3
      \bigm| \cH_0
    \right] \\
    &\quad+
    \frac{1}{\sigma^3}
    \sum_{i=1}^{n+m}
    \sum_{|\kappa| = 3}
    \E \left[
      \left|
      \E \left[ \tilde X_i^\kappa \mid \tilde \cH_{i-1} \right]
      \right|
      \bigm| \cH_0
    \right].
  \end{align*}
  %
  For $1 \leq i \leq n$ we have
  $\|\tilde X_i\| \leq \|X_i\|$
  and $\|\check X_i\| \leq \|V_i^{1/2} \tilde Z_i\|$.
  For $n+1 \leq i \leq n+m$ we have
  $\tilde X_i = H_\tau^{1/2} Z_i / \sqrt m$
  and $\check X_i = H_\tau^{1/2} \tilde Z_i / \sqrt m$
  which are equal in distribution given $\cH_0$.
  Therefore with
  %
  \begin{align*}
    \tilde \beta_{p,k}
    &=
    \sum_{i=1}^{n}
    \E \left[
      \|X_i\|_p \|X_i\|_2^k
      + \|V_i^{1/2} Z_i\|_p \|V_i^{1/2} Z_i\|_2^k
      \bigm| \cH_0
    \right],
  \end{align*}
  %
  we have, since $k \in \{2,3\}$,
  %
  \begin{align*}
    &\sum_{i=1}^{n+m}
    \E \left[
      \|\tilde X_i\|_p \|\tilde X_i\|_2^k
      + \|\check X_i\|_p \|\check X_i\|_2^k
      \bigm| \cH_0
    \right]
    \leq
    \tilde\beta_{p,k}
    + \frac{2}{\sqrt m}
    \E \left[
      \|H_\tau^{1/2} Z\|_p \|H_\tau^{1/2} Z\|_2^k
      \bigm| \cH_0
    \right].
  \end{align*}
  %
  Since $H_i$ is weakly decreasing under the
  semi-definite partial order, we have
  $H_\tau \preceq H_0 = \Sigma + M$
  implying that $|(H_\tau)_{j j}| \leq \|\Sigma + M\|_{\max}$ and
  $\E\big[|(H_\tau^{1/2} Z)_j|^3 \mid \cH_0 \big]
  \leq \sqrt{8/\pi}\, \|\Sigma + M\|_{\max}^{3/2}$.
  Hence as $p \geq 1$ and $k \in \{2,3\}$,
  %
  \begin{align*}
    \E\left[
      \|H_\tau^{1/2}Z\|_p
      \|H_\tau^{1/2}Z\|_2^k
      \bigm| \cH_0
    \right]
    &\leq
    \E\left[\|H_\tau^{1/2} Z\|_1^{k+1}
      \bigm| \cH_0
    \right] \\
    &\leq
    d^{k+1} \max_{1\leq j\leq d}
    \E\left[|(H_\tau^{1/2} Z)_j|^{k+1}
      \bigm| \cH_0
    \right] \\
    &\leq 3 d^4 \,
    \|\Sigma + M\|_{\max}^{(k+1)/2}
    \leq 6 d^4 \,
    \|\Sigma \|_{\max}^{(k+1)/2}
    + 6 d^4 \|M\|.
  \end{align*}
  %
  Assuming some $X_i$ is not identically zero so
  the result is non-trivial,
  and supposing that $\Sigma$ is bounded a.s.\
  (replacing $\Sigma$ by $\Sigma \cdot \I\{\|\Sigma\|_{\max} \leq C\}$
  for an appropriately large $C$ if necessary),
  take $m$ large enough that
  %
  \begin{align}
    \label{eq:bound_extra_terms}
    \frac{2}{\sqrt m}
    \E \left[
      \|H_\tau^{1/2} Z\|_p \|H_\tau^{1/2} Z\|_2^k
      \bigm| \cH_0
    \right]
    \leq
    \frac{1}{4}
    \beta_{p,k}.
  \end{align}
  %
  Further, if $|\kappa| = 3$ then
  $\big|\E \big[
  \tilde X_i^\kappa \mid \tilde \cH_{i-1} \big]\big|
  \leq \big| \E \left[ X_i^\kappa \mid \cH_{i-1} \right]\big|$
  for $1 \leq i \leq n$
  while by symmetry of the Gaussian distribution
  $\E \left[ \tilde X_i^\kappa \mid \tilde \cH_{i-1} \right] = 0$
  for $n+1 \leq i \leq n+m$.
  Hence with
  \begin{align*}
    \tilde \pi_3
    &=
    \sum_{i=1}^{n+m}
    \sum_{|\kappa| = 3}
    \E \Big[ \big|
      \E \left[ X_i^\kappa \mid \cH_{i-1} \right]
    \big| \mid \cH_0 \Big],
  \end{align*}
  %
  we have
  %
  \begin{align*}
    \E\left[
      f\big(\tilde S\big) - f\big(\check S\big)
      \bigm| \cH_0
    \right]
    &\leq
    \min \left\{
      \frac{3 \tilde \beta_{p,2}}{4 \sigma^2 \eta}
      + \frac{\beta_{p,2}}{4 \sigma^2 \eta},
      \frac{3 \tilde \beta_{p,3}}{4 \sigma^3 \eta}
      + \frac{\beta_{p,3}}{4 \sigma^3 \eta}
      + \frac{\tilde \pi_3}{\sigma^3}
    \right\}.
  \end{align*}
  %
  Along with Lemma~\ref{lem:smooth_approximation}, and with
  $\sigma = \eta / t$ and $\varepsilon = \P(\|Z\|_p > t)$,
  we conclude that
  %
  \begin{align*}
    &\P(\tilde S \in A \mid \cH_0)
    =
    \E\big[\I\{\tilde S \in A\} - f(\tilde S)
      \mid \cH_0
    \big]
    + \E\big[f(\tilde S) - f\big(\check S\big)
      \mid \cH_0
    \big]
    + \E \big[f\big(\check S\big)
      \mid \cH_0
    \big] \\
    &\,\leq
    \varepsilon\,\P(\tilde S \in A
    \mid \cH_0)
    + \min \left\{
      \frac{3 \tilde \beta_{p,2}}{4 \sigma^2 \eta}
      + \frac{\beta_{p,2}}{4 \sigma^2 \eta},
      \frac{3 \tilde \beta_{p,3}}{4 \sigma^3 \eta}
      + \frac{\beta_{p,3}}{4 \sigma^3 \eta}
      + \frac{\tilde \pi_3}{\sigma^3}
    \right\} \\
    &\quad+
    \varepsilon
    + (1 - \varepsilon) \P\big(\check S \in A_p^{3\eta}
      \mid \cH_0
    \big) \\
    &\,\leq
    \P\big( \check S \in A_p^{3\eta}
      \mid \cH_0
    \big)
    + 2 \P(\|Z\|_p > t)
    + \min\!\left\{
      \frac{3 \tilde \beta_{p,2} t^2}{4 \eta^3}
      + \frac{\beta_{p,2} t^2}{4 \eta^3},
      \frac{3 \tilde \beta_{p,3} t^3}{4 \eta^4}
      + \frac{\beta_{p,3} t^3}{4 \eta^4}
      + \frac{\tilde \pi_3 t^3}{\eta^3}
    \right\}.
  \end{align*}
  %
  Taking a supremum and an outer expectation yields
  with $\beta_{p,k} = \E\big[\tilde \beta_{p,k}\big]$
  and $\pi_3 = \E[\tilde \pi_3]$,
  %
  \begin{align*}
    &\E^* \left[
      \sup_{A \in \cB(\R^d)}
      \left\{
        \P(\tilde S \in A \mid \cH_0)
        - \P\big( \check S \in A_p^{3\eta} \mid \cH_0 \big)
      \right\}
    \right] \\
    &\quad\leq
    2 \P(\|Z\|_p > t)
    + \min \left\{
      \frac{\beta_{p,2} t^2}{\eta^3},
      \frac{\beta_{p,3} t^3}{\eta^4}
      + \frac{\pi_3 t^3}{\eta^3}
    \right\}.
  \end{align*}
  %
  Finally, since
  $\check S = \sum_{i=1}^n \tilde V_i^{1/2} \tilde Z_i
  \sim \cN(0,\Sigma + M)$ conditional on $\cH_0$,
  the conditional Strassen theorem
  in Lemma~\ref{lem:strassen}
  ensures the existence of $\tilde S$ and
  $\tilde T \mid \cH_0 \sim \cN(0, \Sigma + M)$
  such that
  %
  \begin{align}
    \label{eq:approx_modified_martingale}
    \P\left(\|\tilde S-\tilde T\|_p>3\eta\right)
    &\leq
    \inf_{t>0}
    \left\{
      2 \P(\|Z\|_p > t)
      + \min \left\{
        \frac{\beta_{p,2} t^2}{\eta^3},
        \frac{\beta_{p,3} t^3}{\eta^4} + \frac{\pi_3 t^3}{\eta^3}
      \right\}
    \right\},
  \end{align}
  %
  since the infimum is attained by continuity of $\|Z\|_p$.

  \proofparagraph{conclusion}

  We show how to write
  $\tilde T = (\Sigma + M)^{1/2} W$
  where $W \sim \cN(0,I_d)$
  and use this representation to construct
  $T \mid \cH_0 \sim \cN(0, \Sigma)$.
  By the spectral theorem, let $\Sigma + M = U \Lambda U^\T$
  where $U$ is a $d \times d$ orthogonal random matrix
  and $\Lambda$ is a diagonal $d \times d$ random matrix with
  diagonal entries satisfying
  $\lambda_1 \geq \cdots \geq \lambda_r > 0$
  and $\lambda_{r+1} = \cdots = \lambda_d = 0$
  where $r = \rank (\Sigma + M)$.
  Let $\Lambda^+$ be the Moore--Penrose pseudo-inverse of $\Lambda$
  (obtained by inverting its non-zero elements) and define
  $W = U (\Lambda^+)^{1/2} U^\T \tilde T + U \tilde W$, where
  the first $r$ elements of $\tilde W$ are zero
  and the last $d-r$ elements are i.i.d.\ $\cN(0,1)$
  independent from $\tilde T$.
  Then, it is easy to check that
  $W \sim \cN(0, I_d)$ and that
  $\tilde T = (\Sigma + M)^{1/2} W$.
  Now define $T = \Sigma^{1/2} W$ so
  %
  \begin{equation}%
    \label{eq:approx_target}
    \P\big(\|T - \tilde T\|_p > \eta\big)
    = \P\big(\big\|\big((\Sigma + M)^{1/2}
    - \Sigma^{1/2} \big) W \big\|_p>\eta \big)
    = \delta_p(M, \eta).
  \end{equation}
  %
  Finally
  \eqref{eq:approx_modified_original},
  \eqref{eq:approx_modified_martingale},
  \eqref{eq:approx_target},
  the triangle inequality
  and a union bound conclude the proof since
  by taking an infimum over $M \succeq 0$,
  and by possibly reducing the constant of $1/4$ in
  \eqref{eq:bound_extra_terms} to account for
  this infimum being potentially unattainable,
  %
  \begin{align*}
    \P\big(\|S-T\|_p > 5\eta\big)
    &\leq
    \P\big(\|\tilde S - \tilde T \|_p > 3\eta \big)
    +\P\big(\|S - \tilde S \|_p > \eta\big)
    +\P\big(\|T - \tilde T \|_p > \eta\big) \\
    &\leq
    \inf_{t>0}
    \left\{
      2 \P\big( \|Z\|_p > t \big)
      + \min\left\{
        \frac{\beta_{p,2} t^2}{\eta^3},
        \frac{\beta_{p,3} t^3}{\eta^4}
        + \frac{\pi_3 t^3}{\eta^3}
      \right\}
    \right\} \\
    &\quad+
    \inf_{M \succeq 0}
    \big\{ 2\gamma(M) + \delta_p(M,\eta)
    + \varepsilon_p(M, \eta)\big\}.
  \end{align*}
  %
\end{proof}

Lemma~\ref{lem:sa_martingale} and the martingale approximation
immediately yield Theorem~\ref{thm:sa_dependent}.

\begin{proof}[Theorem~\ref{thm:sa_dependent}]
  Apply Lemma~\ref{lem:sa_martingale} to
  the martingale $\sum_{i=1}^{n} \tilde X_i$,
  noting that $S - \sum_{i=1}^{n} \tilde X_i = U$.
\end{proof}

Bounding the quantities
in Theorem~\ref{thm:sa_dependent} gives a
user-friendly version as Proposition~\ref{pro:sa_simplified}.

\begin{proof}[Proposition~\ref{pro:sa_simplified}]

  We set $M = \nu^2 I_d$ and
  bound each term appearing on the right-hand side of
  the main inequality in Proposition~\ref{pro:sa_simplified}

  \proofparagraph{bounding $\P( \|Z\|_p > t )$}

  By Markov's inequality and Lemma~\ref{lem:gaussian_pnorm},
  we have
  $\P( \|Z\|_p > t ) \leq \E[\|Z\|_p] / t \leq \phi_p(d) / t$.

  \proofparagraph{bounding $\gamma(M)$}

  With $M = \nu^2 I_d$,
  by Markov's inequality,
  $\gamma(M) = \P\big(\Omega \npreceq M\big)
  = \P\big(\|\Omega\|_2 > \nu^2 \big)
  \leq \nu^{-2} \E[\|\Omega\|_2]$.

  \proofparagraph{bounding $\delta(M, \eta)$}

  By Markov's inequality and Lemma~\ref{lem:gaussian_pnorm},
  using
  $\max_j |M_{j j}| \leq \|M\|_2$
  for $M \succeq 0$,
  %
  \begin{align*}
    \delta_{p}(M,\eta)
    &= \P\left(
      \big\|\big((\Sigma +M)^{1/2}- \Sigma^{1/2}\big) Z\big\|_p
      \geq \eta
    \right)
    \leq \frac{\phi_p(d)} {\eta}
    \E \left[
      \big\|(\Sigma +M)^{1/2}- \Sigma^{1/2}\big\|_2
    \right].
  \end{align*}
  %
  For semi-definite matrices
  the eigenvalue operator commutes with smooth matrix functions so
  %
  \begin{align*}
    \|(\Sigma +M)^{1/2}- \Sigma^{1/2}\|_2
    &=
    \max_{1 \leq j \leq d}
    \left|
    \sqrt{\lambda_j(\Sigma) + \nu^2} - \sqrt{\lambda_j(\Sigma)}
    \right|
    \leq \nu
  \end{align*}
  %
  and hence $\delta_{p}(M,\eta) \leq \phi_p(d)\nu / \eta$.

  \proofparagraph{bounding $\varepsilon(M, \eta)$}

  Note that $(M -\Omega)^{1/2}Z$ is a centered Gaussian
  conditional on $\cH_n$,
  on the event $\{\Omega \preceq M\}$.
  We thus have by Markov's inequality,
  Lemma~\ref{lem:gaussian_pnorm}
  and Jensen's inequality that
  %
  \begin{align*}
    \varepsilon_p(M, \eta)
    &= \P\left(\big\| (M - \Omega)^{1/2} Z \big\|_p\geq \eta, \
    \Omega \preceq M\right) \\
    &\leq
    \frac{1}{\eta}
    \E\left[
      \I\{\Omega \preceq M\}
      \E\left[
        \big\| (M - \Omega)^{1/2} Z \big\|_p
        \mid \cH_n
      \right]
    \right] \\
    &\leq
    \frac{\phi_p(d)}{\eta}
    \E\left[
      \I\{\Omega \preceq M\}
      \max_{1 \leq j \leq d}
      \sqrt{(M - \Omega)_{j j}}
    \right]
    \leq
    \frac{\phi_p(d)}{\eta}
    \E\left[
      \sqrt{\|M - \Omega\|_2}
    \right] \\
    &\leq
    \frac{\phi_p(d)}{\eta}
    \E\left[
      \sqrt{\|\Omega\|_2} + \nu
    \right]
    \leq
    \frac{\phi_p(d)}{\eta}
    \left(\sqrt{\E[\|\Omega\|_2]} + \nu \right).
  \end{align*}
  %
  Thus by Theorem~\ref{thm:sa_dependent} and the previous parts,
  %
  \begin{align*}
    \P\big(\|S-T\|_p > 6\eta\big)
    &\leq
    \inf_{t>0}
    \left\{
      2 \P\big(\|Z\|_p>t\big)
      + \min\left\{
        \frac{\beta_{p,2} t^2}{\eta^3},
        \frac{\beta_{p,3} t^3}{\eta^4}
        + \frac{\pi_3 t^3}{\eta^3}
      \right\}
    \right\} \\
    &\quad+
    \inf_{M \succeq 0}
    \big\{ 2\gamma(M) + \delta_p(M,\eta)
    + \varepsilon_p(M, \eta)\big\}
    +\P\big(\|U\|_p>\eta\big) \\
    &\leq
    \inf_{t>0}
    \left\{
      \frac{2 \phi_p(d)}{t}
      + \min\left\{
        \frac{\beta_{p,2} t^2}{\eta^3},
        \frac{\beta_{p,3} t^3}{\eta^4}
        + \frac{\pi_3 t^3}{\eta^3}
      \right\}
    \right\} \\
    &\quad+
    \inf_{\nu > 0}
    \left\{ \frac{2\E \left[ \|\Omega\|_2 \right]}{\nu^2}
      + \frac{2 \phi_p(d) \nu}{\eta}
    \right\}
    + \frac{\phi_p(d) \sqrt{\E \left[ \|\Omega\|_2 \right]}}{\eta}
    +\P\big(\|U\|_p>\eta\big).
  \end{align*}
  %
  Set $t = 2^{1/3} \phi_p(d)^{1/3} \beta_{p,2}^{-1/3} \eta$
  and $\nu = \E[\|\Omega\|_2]^{1/3} \phi_p(d)^{-1/3} \eta^{1/3}$,
  then replace $\eta$ with $\eta / 6$ to see
  %
  \begin{align*}
    \P\big(\|S-T\|_p > 6\eta\big)
    &\leq
    24 \left(
      \frac{\beta_{p,2} \phi_p(d)^2}{\eta^3}
    \right)^{1/3}
    + 17 \left(
      \frac{\E \left[ \|\Omega\|_2 \right] \phi_p(d)^2}{\eta^2}
    \right)^{1/3}
    +\P\left(\|U\|_p>\frac{\eta}{6}\right).
  \end{align*}
  %
  Whenever $\pi_3 = 0$ we can set
  $t = 2^{1/4} \phi_p(d)^{1/4} \beta_{p,3}^{-1/4} \eta$,
  and with $\nu$ as above we obtain
  %
  \begin{align*}
    \P\big(\|S-T\|_p > \eta\big)
    &\leq
    24 \left(
      \frac{\beta_{p,3} \phi_p(d)^3}{\eta^4}
    \right)^{1/4}
    + 17 \left(
      \frac{\E \left[ \|\Omega\|_2 \right] \phi_p(d)^2}{\eta^2}
    \right)^{1/3}
    +\P\left(\|U\|_p>\frac{\eta}{6}\right).
  \end{align*}
  %
\end{proof}

After establishing Proposition~\ref{pro:sa_simplified},
Corollaries~\ref{cor:sa_mixingale}, \ref{cor:sa_martingale}
and \ref{cor:sa_indep} follow easily, as shown in the main text.

\begin{proof}[Corollary~\ref{cor:sa_mixingale}]
  Proposition~\ref{pro:sa_simplified} with
  $\P ( \|U\|_p > \frac{\eta}{6} )
  \leq \frac{6}{\eta} \sum_{i=1}^{n} c_i (\zeta_{i} + \zeta_{n-i+1})$.
\end{proof}

\begin{proof}[Corollary~\ref{cor:sa_martingale}]
  By Proposition~\ref{pro:sa_simplified}
  with $U=0$ a.s.
\end{proof}

\begin{proof}[Corollary~\ref{cor:sa_indep}]
  By Corollary~\ref{cor:sa_martingale}
  with $\Omega=0$ a.s.
\end{proof}

We conclude this section with a discussion expanding on the comments made
in Remark~\ref{rem:coupling_bounds_probability} on deriving bounds in
probability from Yurinskii's coupling. Consider for illustration the
independent data second-order result given in
Corollary~\ref{cor:sa_indep}: for each $\eta > 0$,
there exists $T_n \mid \cH_0 \sim \cN(0, \Sigma)$ satisfying
%
\begin{align*}
  \P\big(\|S_n-T_n\|_p > \eta\big)
  &\leq
  24 \left(
    \frac{\beta_{p,2} \phi_p(d)^2}{\eta^3}
  \right)^{1/3},
\end{align*}
%
where here we make explicit the dependence on the sample size $n$ for clarity.
The naive approach to converting this into a probability bound for
$\|S_n-T_n\|_p$ is to select $\eta$ to ensure the right-hand side is
of order $1$, arguing that the probability can then be made arbitrarily
small by taking, in this case, $\eta$ to be a large enough multiple of
$\beta_{p,2}^{1/3} \phi_p(d)^{2/3}$. However, the somewhat subtle mistake is
in neglecting the fact that the realization of the coupling variable $T_n$
will in general depend on $\eta$, rendering the resulting
bound invalid.
As an explicit example of this phenomenon, take $\eta > 1$ and suppose
$\|S_n - T_n(\eta)\| = \eta$ with probability $1 - 1/\eta$ and
$\|S_n - T_n(\eta)\| = n$ with probability $1/\eta$.
Then $\P\big(\|S_n - T_n(\eta)\| > \eta\big) = 1/\eta$
but it is not true for any $\eta$ that $\|S_n - T_n(\eta)\| \lesssim_\P 1$.

We propose in Remark~\ref{rem:coupling_bounds_probability} the following fix.
Instead of selecting $\eta$ to ensure the right-hand side is of order $1$,
we instead choose it so the bound converges (slowly) to zero. This is
easily achieved by taking the naive and incorrect bound and multiplying
by some divergent sequence $R_n$. The resulting inequality reads,
in the case of Corollary~\ref{cor:sa_indep} with
$\eta = \beta_{p,2}^{1/3} \phi_p(d)^{2/3} R_n$,
%
\begin{align*}
  \P\Big(\|S_n-T_n\|_p >
    \beta_{p,2}^{1/3} \phi_p(d)^{2/3} R_n
  \Big)
  &\leq
  \frac{24}{R_n}
  \to 0.
\end{align*}
%
We thus recover, for the price of a rate which is slower by an arbitrarily
small amount, a valid upper bound in probability, as we can immediately
conclude that
%
\begin{align*}
  \|S_n-T_n\|_p
  \lesssim_\P
  \beta_{p,2}^{1/3} \phi_p(d)^{2/3} R_n.
\end{align*}

\subsection{Strong approximation for martingale empirical processes}

We begin by presenting some calculations omitted from the main text
relating to the motivating example of kernel density estimation with
i.i.d.\ data.
First, the bias of this estimator is bounded as
%
\begin{align*}
  \big| \E \big[ \hat g(x) \big] - g(x) \big|
  &=
  \left|
  \int_{\frac{-x}{h}}^{\frac{1-x}{h}}
  K(\xi)
  \diff \xi
  - 1
  \right|
  \leq
  2 \int_{\frac{a}{h}}^\infty
  \frac{1}{\sqrt{2 \pi}}
  e^{-\frac{\xi^2}{2}}
  \diff \xi
  \leq
  \frac{h}{a}
  \sqrt{\frac{2}{\pi}}
  e^{-\frac{a^2}{2 h^2}}.
\end{align*}
%
Next, we do the calculations necessary to apply
Corollary~\ref{cor:sa_indep}.
Define
$k_{i j} = \frac{1}{n h} K \left( \frac{X_i - x_j}{h} \right)$ and
$k_i = (k_{i j} : 1 \leq j \leq N)$.
Then $\|k_i\|_\infty \leq \frac{1}{n h \sqrt{2 \pi}}$ a.s.\ and
$\E[\|k_i\|_2^2] \leq \frac{N}{n^2 h} \int_{-\infty}^\infty K(\xi)^2 \diff \xi
\leq \frac{N}{2 n^2 h \sqrt{\pi}}$.
Let $V = \Var[k_i] \in \R^{N \times N}$,
so assuming that $1/h \geq \log 2 N$,
by Lemma~\ref{lem:gaussian_useful} we bound
%
\begin{align*}
  \beta_{\infty,2}
  &=
  n \E\left[\| k_i \|^2_2 \| k_i \|_\infty
  \right]
  + n \E \left[ \|V^{1/2} Z \|^2_2 \|V^{1/2} Z \|_\infty \right] \\
  &\leq
  \frac{N}{\sqrt{8} n^2 h^2 \pi}
  + \frac{4 N \sqrt{\log 2 N}}{\sqrt{8} n^2 h^{3/2} \pi^{3/4}}
  \leq
  \frac{N}{n^2 h^2}.
\end{align*}
%
Finally, we verify the stochastic continuity bounds.
By the Lipschitz property of $K$, it is easy to show that
for $x,x' \in \cX$ we have
$\left|\frac{1}{h} K \left( \frac{X_i - x}{h} \right)
- \frac{1}{h} K \left( \frac{X_i - x'}{h} \right)\right|
\lesssim \frac{|x-x'|}{h^2}$ almost surely, and also that
$\E \Big[ \left|\frac{1}{h} K \left( \frac{X_i - x}{h} \right)
- \frac{1}{h} K \left( \frac{X_i - x'}{h} \right)\right|^2 \Big]
\lesssim \frac{|x-x'|^2}{h^3}$.
By chaining with the Bernstein--Orlicz norm and polynomial covering numbers,
%
\begin{align*}
  \sup_{|x-x'| \leq \delta}
  \big\|S(x) - S(x')\big\|_\infty
  \lesssim_\P
  \delta
  \sqrt{\frac{\log n}{n h^3}}
\end{align*}
%
whenever $\log(N/h) \lesssim \log n$
and $n h \gtrsim \log n$.
By a Gaussian process maximal inequality
\citep[Corollary~2.2.8]{van1996weak}
the same bound holds for $T(x)$ with
%
\begin{align*}
  \sup_{|x-x'| \leq \delta}
  \big\|T(x) - T(x')\big\|_\infty
  \lesssim_\P
  \delta
  \sqrt{\frac{\log n}{n h^3}}.
\end{align*}

\begin{proof}[Lemma~\ref{lem:kde_eigenvalue}]

  For $x, x' \in [a, 1-a]$, the scaled covariance function
  of this nonparametric estimator is
  %
  \begin{align*}
    n h\, \Cov\big[\hat g(x), \hat g(x')\big]
    &=
    \frac{1}{h}
    \E \left[
      K \left( \frac{X_i - x}{h} \right)
      K \left( \frac{X_i - x'}{h} \right)
    \right] \\
    &\quad-
    \frac{1}{h}
    \E \left[
      K \left( \frac{X_i - x}{h} \right)
    \right]
    \E \left[
      K \left( \frac{X_i - x'}{h} \right)
    \right] \\
    &=
    \frac{1}{2 \pi}
    \int_{\frac{-x}{h}}^{\frac{1-x}{h}}
    \exp \left( - \frac{t^2}{2} \right)
    \exp \left( - \frac{1}{2} \left( t + \frac{x - x'}{h} \right)^2 \right)
    \diff t
    - h I(x) I(x')
  \end{align*}
  %
  where
  $I(x) = \frac{1}{\sqrt 2 \pi} \int_{-x/h}^{(1-x)/h} e^{-t^2/2} \diff t$.
  Completing the square and a substitution gives
  %
  \begin{align*}
    n h\, \Cov\big[\hat g(x), \hat g(x')\big]
    &=
    \frac{1}{2 \pi}
    \exp \left( - \frac{1}{4} \left( \frac{x-x'}{h} \right)^2 \right)
    \int_{\frac{-x-x'}{2h}}^{\frac{2-x-x'}{2h}}
    \exp \left(-t^2\right)
    \diff t
    - h I(x) I(x').
  \end{align*}
  %
  Now we show that since $x, x'$ are not too close to the boundary
  of $[0,1]$,
  the limits in the above integral can be replaced by $\pm \infty$.
  Note that $\frac{-x-x'}{2h} \leq \frac{-a}{h}$
  and $\frac{2-x-x'}{2h} \geq \frac{a}{h}$ so
  %
  \begin{align*}
    \int_{-\infty}^{\infty}
    \exp \left(-t^2\right)
    \diff t
    - \int_{\frac{-x-x'}{2h}}^{\frac{2-x-x'}{2h}}
    \exp \left(-t^2\right)
    \diff t
    \leq
    2 \int_{a/h}^\infty
    \exp \left(-t^2\right)
    \diff t
    \leq
    \frac{h}{a}
    \exp \left(- \frac{a^2}{h^2}\right).
  \end{align*}
  %
  Therefore since
  $\int_{-\infty}^{\infty} e^{-t^2} \diff t = \sqrt \pi$,
  %
  \begin{align*}
    \left|
    n h\, \Cov\big[\hat g(x), \hat g(x')\big]
    - \frac{1}{2 \sqrt \pi}
    \exp \left( - \frac{1}{4} \left( \frac{x-x'}{h} \right)^2 \right)
    + h I(x) I(x')
    \right|
    \leq
    \frac{h}{2 \pi a}
    \exp \left(- \frac{a^2}{h^2}\right).
  \end{align*}
  %
  Define the $N \times N$ matrix
  $\tilde\Sigma_{i j} = \frac{1}{2 \sqrt \pi}
  \exp \left( - \frac{1}{4} \left( \frac{x_i-x_j}{h} \right)^2 \right)$.
  By \citet[Proposition~2.4,
  Proposition~2.5 and Equation~2.10]{baxter1994norm},
  with
  $\cB_k = \big\{b \in \R^\Z :
  \sum_{i \in \Z} \I\{b_i \neq 0\} \leq k \big\}$,
  %
  \begin{align*}
    \inf_{k \in \N}
    \inf_{b \in \R^k}
    \frac{\sum_{i=1}^k \sum_{j=1}^k b_i b_j \, e^{-\lambda(i-j)^2}}
    {\sum_{i=1}^k b_i^2}
    =
    \sqrt{\frac{\pi}{\lambda}}
    \sum_{i=-\infty}^{\infty}
    \exp \left( - \frac{(\pi e + 2 \pi i)^2}{4 \lambda} \right).
  \end{align*}
  %
  We use Riemann sums,
  noting that $\pi e + 2 \pi x = 0$ at
  $x = -e/2 \approx -1.359$.
  Consider the substitutions
  $\Z \cap (-\infty, -3] \mapsto (-\infty, -2]$,
  $\{-2, -1\} \mapsto \{-2, -1\}$ and
  $\Z \cap [0, \infty) \mapsto [-1, \infty)$.
  %
  \begin{align*}
    \sum_{i \in \Z}
    e^{-(\pi e + 2 \pi i)^2 / 4 \lambda}
    &\leq
    \int_{-\infty}^{-2}
    e^{ - (\pi e + 2 \pi x)^2/4 \lambda}
    \diff x
    + e^{- (\pi e - 4 \pi)^2/4 \lambda} \\
    &\quad+
    e^{ - (\pi e - 2 \pi)^2 / 4 \lambda}
    + \hspace*{-1mm} \int_{-1}^{\infty}
    e^{ -(\pi e + 2 \pi x)^2 / 4 \lambda}
    \diff x.
  \end{align*}
  %
  Now use the substitution $t = \frac{\pi e + 2 \pi x}{2 \sqrt \lambda}$
  and suppose $\lambda < 1$, yielding
  %
  \begin{align*}
    \sum_{i \in \Z}
    e^{-(\pi e + 2 \pi i)^2 / 4 \lambda}
    &\leq
    \frac{\sqrt \lambda}{\pi}
    \int_{-\infty}^{\frac{\pi e - 4 \pi}{2 \sqrt \lambda}}
    e^{-t^2}
    \diff t
    + e^{- (\pi e - 4 \pi)^2/4 \lambda} \\
    &\quad+
    e^{ - (\pi e - 2 \pi)^2 / 4 \lambda}
    + \frac{\sqrt \lambda}{\pi}
    \int_{\frac{\pi e - 2 \pi}{2 \sqrt \lambda}}^{\infty}
    e^{-t^2}
    \diff t \\
    &\leq
    \left( 1 + \frac{1}{\pi} \frac{\lambda}{4 \pi - \pi e} \right)
    e^{-(\pi e - 4 \pi)^2 / 4 \lambda}
    +
    \left( 1 + \frac{1}{\pi} \frac{\lambda}{\pi e - 2 \pi} \right)
    e^{- (\pi e - 2 \pi)^2 / 4 \lambda} \\
    &\leq
    \frac{13}{12}
    e^{-(\pi e - 4 \pi)^2 / 4 \lambda}
    +
    \frac{8}{7}
    e^{- (\pi e - 2 \pi)^2 / 4 \lambda}
    \leq
    \frac{9}{4}
    \exp \left( - \frac{5}{4 \lambda} \right).
  \end{align*}
  %
  Therefore
  %
  \begin{align*}
    \inf_{k \in \N}
    \inf_{b \in \cB_k}
    \frac{\sum_{i \in \Z} \sum_{j \in \Z} b_i b_j \, e^{-\lambda(i-j)^2}}
    {\sum_{i \in \Z} b_i^2}
    < \frac{4}{\sqrt \lambda}
    \exp \left( - \frac{5}{4 \lambda} \right)
    < 4 e^{-1/\lambda}.
  \end{align*}
  %
  From this and since
  $\tilde\Sigma_{i j} = \frac{1}{2 \sqrt \pi} e^{-\lambda(i-j)^2}$
  with $\lambda = \frac{1}{4(N-1)^2 h^2} \leq \frac{\delta^2}{h^2}$,
  for each $h$ and some $\delta \leq h$,
  %
  \begin{align*}
    \lambda_{\min}(\tilde\Sigma)
    &\leq
    2 e^{-h^2/\delta^2}.
  \end{align*}
  %
  Recall that
  %
  \begin{align*}
    \left|
    \Sigma_{i j}
    - \tilde\Sigma_{i j}
    + h I(x_i) I(x_j)
    \right|
    \leq
    \frac{h}{2 \pi a}
    \exp \left(- \frac{a^2}{h^2}\right).
  \end{align*}
  %
  Now for any positive semi-definite $N \times N$ matrices $A$ and $B$
  and vector $v$ we have $\lambda_{\min}(A - v v^\T) \leq \lambda_{\min}(A)$
  and $\lambda_{\min}(B) \leq \lambda_{\min}(A) + \|B-A\|_2
  \leq \lambda_{\min}(A) + N \|B-A\|_{\max}$.
  Hence with $I_i = I(x_i)$,
  %
  \begin{align*}
    \lambda_{\min}(\Sigma)
    &\leq
    \lambda_{\min}(\tilde\Sigma - h I I^\T)
    + \frac{N h}{2 \pi a}
    \exp \left(- \frac{a^2}{h^2}\right)
    \leq
    2 e^{-h^2/\delta^2}
    + \frac{h}{\pi a \delta}
    e^{-a^2 / h^2}.
  \end{align*}
\end{proof}

\begin{proof}[Proposition~\ref{pro:emp_proc}]

  Let $\cF_\delta$ be a $\delta$-cover of $(\cF, d)$.
  Using a union bound, we can write
  %
  \begin{align*}
    &\P\left(\sup_{f \in \cF}
      \big| S(f) - T(f) \big|
    \geq 2t + \eta \right)
    \leq
    \P\left(\sup_{f \in \cF_\delta}
      \big| S(f) - T(f) \big|
    \geq \eta \right) \\
    &\qquad\qquad+
    \P\left(\sup_{d(f,f') \leq \delta}
      \big| S(f) - S(f') \big|
    \geq t \right)
    + \P\left(\sup_{d(f,f') \leq \delta}
      \big| T(f) - T(f') \big|
    \geq t \right).
  \end{align*}

  \proofparagraph{bounding the difference on $\cF_\delta$}

  We apply Corollary~\ref{cor:sa_martingale}
  with $p = \infty$ to the
  martingale difference sequence
  $\cF_\delta(X_i) = \big(f(X_i) : f \in \cF_\delta\big)$
  which takes values in $\R^{|\cF_\delta|}$.
  Square integrability can be assumed otherwise
  $\beta_\delta = \infty$.
  Note
  $\sum_{i=1}^n \cF_\delta(X_i) = S(\cF_\delta)$
  and $\phi_\infty(\cF_\delta) \leq \sqrt{2 \log 2 |\cF_\delta|}$.
  Therefore there exists a conditionally Gaussian vector $T(\cF_\delta)$
  with the same covariance structure as $S(\cF_\delta)$
  conditional on $\cH_0$ satisfying
  %
  \begin{align*}
    \P\left(
      \sup_{f \in \cF_\delta}
      \big| S(f) - T(f) \big|
      \geq \eta
    \right)
    &\leq
    \frac{24\beta_\delta^{\frac{1}{3}}
    (2\log 2 |\cF_\delta|)^{\frac{1}{3}}}{\eta}
    + 17\left(\frac{\sqrt{2 \log 2 |\cF_\delta|}
    \sqrt{\E\left[\|\Omega_\delta\|_2\right]}}{\eta }\right)^{\frac{2}{3}}.
  \end{align*}

  \proofparagraph{bounding the fluctuations in $S(f)$}

  Since $\big\| S(f) - S(f') \big\|_\psi \leq L d(f,f')$,
  by Theorem~2.2.4 in \citet{van1996weak}
  %
  \begin{align*}
    \left\|
    \sup_{d(f,f') \leq \delta}
    \big| S(f) - S(f') \big|
    \right\|_\psi
    &\leq
    C_\psi L
    \left(
      \int_0^\delta
      \psi^{-1}(N_\varepsilon) \diff{\varepsilon}
      + \delta \psi^{-1}(N_\delta^2)
    \right)
    = C_\psi L J_\psi(\delta).
  \end{align*}
  %
  Then, by Markov's inequality and the definition of the Orlicz norm,
  %
  \begin{align*}
    \P\left(
      \sup_{d(f,f') \leq \delta}
      \big| S(f) - S(f') \big|
      \geq t
    \right)
    &\leq
    \psi\left(\frac{t}{C_\psi L J_\psi(\delta)} \right)^{-1}.
  \end{align*}

  \proofparagraph{bounding the fluctuations in $T(f)$}

  By the Vorob'ev--Berkes--Philipp theorem
  \citep{dudley1999uniform},
  $T(\cF_\delta)$ extends to a conditionally Gaussian process $T(f)$.
  Firstly since
  $\bigvvvert T(f) - T(f') \bigvvvert_2 \leq L d(f,f')$
  conditionally on $\cH_0$,
  and $T(f)$ is a conditional Gaussian process, we have
  $\big\| T(f) - T(f') \big\|_{\psi_2} \leq 2 L d(f,f')$
  conditional on $\cH_0$
  by \citet[Chapter~2.2, Complement~1]{van1996weak},
  where $\psi_2(x) = \exp(x^2) - 1$.
  Thus again by Theorem~2.2.4 in \citet{van1996weak},
  again conditioning on $\cH_0$,
  %
  \begin{align*}
    \left\|
    \sup_{d(f,f') \leq \delta}
    \big| T(f) - T(f') \big|
    \right\|_{\psi_2}
    &\leq
    C_1 L
    \int_0^\delta
    \sqrt{\log N_\varepsilon} \diff{\varepsilon}
    = C_1 L J_2(\delta)
  \end{align*}
  %
  for some universal constant $C_1 > 0$,
  where we used $\psi_2^{-1}(x) = \sqrt{\log(1+x)}$
  and monotonicity of covering numbers.
  Then by Markov's inequality and the definition of the Orlicz norm,
  %
  \begin{align*}
    \P\left(
      \sup_{d(f,f') \leq \delta}
      \big| T(f) - T(f') \big|
      \geq t
    \right)
    &\leq
    \left(
      \exp\left(
        \frac{t^2}{C_1^2 L^2 J_2(\delta)^2}
      \right) - 1
    \right)^{-1}
    \vee 1 \\
    &\leq
    2 \exp\left(
      \frac{-t^2}{C_1^2 L^2 J_2(\delta)^2}
    \right).
  \end{align*}
  %

  \proofparagraph{conclusion}

  The result follows by scaling $t$ and $\eta$
  and enlarging constants if necessary.
  %
\end{proof}

\subsection{Applications to nonparametric regression}

\begin{proof}[Proposition~\ref{pro:series}]

  We proceed according to the decomposition given in
  Section~\ref{sec:series}.
  By stationarity and Lemma~SA-2.1 in
  \citet{cattaneo2020large},
  we have $\sup_w \|p(w)\|_1 \lesssim 1$
  and also $\|H\|_1 \lesssim n/k$
  and $\|H^{-1}\|_1 \lesssim k/n$.

  \proofparagraph{bounding $\beta_{\infty,2}$ and $\beta_{\infty,3}$}

  Set $X_i = p(W_i) \varepsilon_i$
  so $S = \sum_{i=1}^n X_i$
  and set $\sigma^2_i = \sigma^2(W_i)$ and
  $V_i = \Var[X_i \mid \cH_{i-1}]
  = \sigma_i^2 p(W_i) p(W_i)^\T$.
  Recall from
  Corollary~\ref{cor:sa_martingale} that for $r \in \{2,3\}$,
  %
  \begin{align*}
    \beta_{\infty,r}
    = \sum_{i=1}^n \E\left[\| X_i \|^r_2 \| X_i \|_\infty
    + \|V_i^{1/2} Z_i \|^r_2 \|V_i^{1/2} Z_i \|_\infty \right]
  \end{align*}
  %
  with $Z_i \sim \cN(0,1)$ i.i.d.\ and independent of $V_i$.
  For the first term, we use
  $\sup_w \|p(w)\|_2 \lesssim 1$
  and bounded third moments of $\varepsilon_i$:
  %
  \begin{align*}
    \E\left[ \| X_i \|^r_2 \| X_i \|_\infty \right]
    &\leq
    \E\left[ |\varepsilon_i|^3 \| p(W_i) \|^{r+1}_2 \right]
    \lesssim 1.
  \end{align*}
  %
  For the second term, apply Lemma~\ref{lem:gaussian_useful} conditionally on
  $\cH_n$ with $\sup_w \|p(w)\|_2 \lesssim 1$ to see
  %
  \begin{align*}
    \E\left[ \|V_i^{1/2} Z_i \|^r_2 \|V_i^{1/2} Z_i \|_\infty \right]
    &\lesssim
    \sqrt{\log 2k} \
    \E\left[
      \max_{1 \leq j \leq k}
      (V_i)_{j j}^{1/2}
      \bigg( \sum_{j=1}^k (V_i)_{j j} \bigg)^{r/2}
    \right] \\
    &\lesssim
    \sqrt{\log 2k} \
    \E\left[
      \sigma_i^{r+1}
      \max_{1 \leq j \leq k}
      p(W_i)_j
      \bigg(
        \sum_{j=1}^k
        p(W_i)_{j}^2
      \bigg)^{r/2}
    \right] \\
    &\lesssim
    \sqrt{\log 2k} \
    \E\left[
      \sigma_i^{r+1}
    \right]
    \lesssim
    \sqrt{\log 2k}.
  \end{align*}
  %
  Putting these together yields
  %
  $\beta_{\infty,2} \lesssim n \sqrt{\log 2k}$
  and $\beta_{\infty,3} \lesssim n \sqrt{\log 2k}$.

  \proofparagraph{bounding $\Omega$}

  Set $\Omega = \sum_{i=1}^n \big(V_i - \E[V_i] \big)$ as in
  Lemma~\ref{lem:sa_martingale} so
  %
  \begin{align*}
    \Omega
    &= \sum_{i=1}^n
    \big(\sigma_i^2 p(W_i)p(W_i)^\T - \E\left[ \sigma_i^2 p(W_i)p(W_i)^\T
    \right]\big).
  \end{align*}
  %
  Observe that $\Omega_{j l}$ is the sum of a zero-mean
  strictly stationary $\alpha$-mixing sequence and so $\E[\Omega_{j l}^2]
  \lesssim n$ by
  Lemma~\ref{lem:variance_mixing}\ref{eq:variance_mixing_bounded}.
  Since the basis functions
  satisfy Assumption~3 in \citet{cattaneo2020large}, $\Omega$ has a bounded
  number of non-zero entries in each row, and so by Jensen's inequality
  %
  \begin{align*}
    \E\left[
      \|\Omega\|_2
    \right]
    &\leq
    \E\left[
      \|\Omega\|_\rF
    \right]
    \leq
    \left(
      \sum_{j=1}^k
      \sum_{l=1}^k
      \E\left[
        \Omega_{j l}^2
      \right]
    \right)^{1/2}
    \lesssim \sqrt{n k}.
  \end{align*}
  %

  \proofparagraph{strong approximation}

  By Corollary~\ref{cor:sa_martingale} and the previous parts,
  with any sequence $R_n \to \infty$,
  %
  \begin{align*}
    \|S  - T \|_\infty
    &\lesssim_\P
    \beta_{\infty,2}^{1/3} (\log 2k)^{1/3} R_n
    + \sqrt{\log 2k} \sqrt{\E[\|\Omega\|_2]} R_n \\
    &\lesssim_\P
    n^{1/3} \sqrt{\log 2k} R_n
    + (n k)^{1/4} \sqrt{\log 2k} R_n.
  \end{align*}
  %
  If further $\E \left[ \varepsilon_i^3 \mid \cH_{i-1} \right] = 0$ then
  the third-order version of Corollary~\ref{cor:sa_martingale}
  applies since
  %
  \begin{align*}
    \pi_3
    &=
    \sum_{i=1}^{n}
    \sum_{|\kappa| = 3}
    \E \Big[ \big|
      \E [ X_i^\kappa \mid \cH_{i-1} ]
    \big| \Big]
    =
    \sum_{i=1}^{n}
    \sum_{|\kappa| = 3}
    \E \Big[ \big|
      p(W_i)^\kappa \,
      \E [ \varepsilon_i^3 \mid \cH_{i-1} ]
    \big| \Big]
    = 0,
  \end{align*}
  %
  giving
  %
  \begin{align*}
    \|S  - T \|_\infty
    &\lesssim_\P
    \beta_{\infty,3}^{1/4} (\log 2k)^{3/8} R_n
    + \sqrt{\log 2k} \sqrt{\E[\|\Omega\|_2]} R_n
    \lesssim_\P
    (n k)^{1/4} \sqrt{\log 2k} R_n.
  \end{align*}
  %
  By H{\"o}lder's inequality and with
  $\|H^{-1}\|_1 \lesssim k/n$ we have
  %
  \begin{align*}
    \sup_{w \in \cW}
    \left|
    p(w)^\T H^{-1} S
    - p(w)^\T H^{-1} T
    \right|
    &\leq
    \sup_{w \in \cW}
    \|p(w)\|_1
    \|H^{-1}\|_1
    \| S - T \|_\infty
    \lesssim
    n^{-1} k
    \| S - T \|_\infty.
  \end{align*}

  \proofparagraph{convergence of $\hat H$}

  We have
  $\hat H - H = \sum_{i=1}^n \big(p(W_i)p(W_i)^\T - \E\left[
  p(W_i)p(W_i)^\T \right]\big)$.
  Observe that $(\hat H - H)_{j l}$ is the sum of
  a zero-mean strictly stationary $\alpha$-mixing sequence and so
  $\E[(\hat H - H)_{j l}^2] \lesssim n$ by
  Lemma~\ref{lem:variance_mixing}\ref{eq:variance_mixing_bounded}.
  Since the basis
  functions satisfy Assumption~3 in \citet{cattaneo2020large},
  $\hat H-H$ has a
  bounded number of non-zero entries in each row and so by Jensen's inequality
  %
  \begin{align*}
    \E\left[
      \|\hat H-H\|_1
    \right]
    &=
    \E\left[
      \max_{1 \leq i \leq k}
      \sum_{j=1}^k
      \big|(\hat H-H)_{i j}\big|
    \right]
    \leq
    \E\left[
      \sum_{1 \leq i \leq k}
      \Bigg(
        \sum_{j=1}^k
        |(\hat H-H)_{i j}|
      \Bigg)^2
    \right]^{\frac{1}{2}}
    \lesssim \sqrt{n k}.
  \end{align*}

  \proofparagraph{bounding the matrix term}

  Note $\|\hat H^{-1}\|_1 \leq \|H^{-1}\|_1
  + \|\hat H^{-1}\|_1 \|\hat H-H\|_1 \|H^{-1}\|_1$
  so by the previous part, we deduce
  %
  \begin{align*}
    \|\hat H^{-1}\|_1
    \leq
    \frac{\|H^{-1}\|_1}
    {1 - \|\hat H-H\|_1 \|H^{-1}\|_1}
    \lesssim_\P
    \frac{k/n}
    {1 - \sqrt{n k}\, k/n}
    \lesssim_\P
    \frac{k}{n}
  \end{align*}
  %
  as $k^3 / n \to 0$. Note that by the martingale structure, since
  $p(W_i)$ is bounded and supported on a region with volume at most of the order
  $1/k$, and as $W_i$ has a Lebesgue density,
  %
  \begin{align*}
    \Var[T_j]
    &=
    \Var[S_j]
    =
    \Var\left[
      \sum_{i=1}^n \varepsilon_i p(W_i)_j
    \right]
    =
    \sum_{i=1}^n
    \E\left[
      \sigma_i^2 p(W_i)_j^2
    \right]
    \lesssim
    \frac{n}{k}.
  \end{align*}
  %
  So by the Gaussian maximal inequality in Lemma~\ref{lem:gaussian_pnorm},
  $\|T\|_\infty \lesssim_\P \sqrt{\frac{n \log 2k}{k}}$.
  Since $k^3/n \to 0$,
  %
  \begin{align*}
    \sup_{w \in \cW}
    \left|
    p(w)^\T (\hat H^{-1} - H^{-1}) S
    \right|
    &\leq
    \sup_{w \in \cW}
    \|p(w)^\T\|_1
    \|\hat H^{-1}\|_1
    \|\hat H - H\|_1
    \|H^{-1}\|_1
    \|S - T\|_\infty \\
    &\quad+
    \sup_{w \in \cW}
    \|p(w)^\T\|_1
    \|\hat H^{-1}\|_1
    \|\hat H - H\|_1
    \|H^{-1}\|_1
    \|T\|_\infty \\
    &\lesssim_\P
    \frac{k}{n}
    \sqrt{n k}
    \frac{k}{n}
    \left(
      n^{1/3} \sqrt{\log 2k}
      + (n k)^{1/4} \sqrt{\log 2k}
    \right) \\
    &\quad+
    \frac{k}{n}
    \sqrt{n k}
    \frac{k}{n}
    \sqrt{\frac{n \log 2k}{k}} \\
    &\lesssim_\P
    \frac{k^2}{n}
    \sqrt{\log 2k}.
  \end{align*}
  %

  \proofparagraph{conclusion of the main result}

  By the previous parts,
  with $G(w) = p(w)^\T H^{-1} T$,
  %
  \begin{align*}
    &\sup_{w \in \cW}
    \left|
    \hat\mu(w) - \mu(w)
    - p(w)^\T H^{-1} T
    \right| \\
    &\quad=
    \sup_{w \in \cW}
    \left|
    p(w)^\T H^{-1} (S - T)
    + p(w)^\T (\hat H^{-1} - H^{-1}) S
    + \Bias(w)
    \right| \\
    &\quad\lesssim_\P
    \frac{k}{n}
    \|S - T\|_\infty
    + \frac{k^2}{n} \sqrt{\log 2k}
    + \sup_{w \in \cW} |\Bias(w)| \\
    &\quad\lesssim_\P
    \frac{k}{n}
    \left( n^{1/3} \sqrt{\log 2k} + (n k)^{1/4} \sqrt{\log 2k} \right) R_n
    + \frac{k^2}{n} \sqrt{\log 2k}
    + \sup_{w \in \cW} |\Bias(w)| \\
    &\quad\lesssim_\P
    n^{-2/3} k \sqrt{\log 2k} R_n
    + n^{-3/4} k^{5/4} \sqrt{\log 2k} R_n
    + \frac{k^2}{n} \sqrt{\log 2k}
    + \sup_{w \in \cW} |\Bias(w)| \\
    &\quad\lesssim_\P
    n^{-2/3} k \sqrt{\log 2k} R_n
    + \sup_{w \in \cW} |\Bias(w)|
  \end{align*}
  %
  since $k^3/n \to 0$.
  If further $\E \left[ \varepsilon_i^3 \mid \cH_{i-1} \right] = 0$ then
  %
  \begin{align*}
    \sup_{w \in \cW}
    \left|
    \hat\mu(w) - \mu(w)
    - p(w)^\T H^{-1} T
    \right|
    &\lesssim_\P
    \frac{k}{n}
    \|S - T\|_\infty
    + \frac{k^2}{n} \sqrt{\log 2k}
    + \sup_{w \in \cW} |\Bias(w)| \\
    &\lesssim_\P
    n^{-3/4} k^{5/4} \sqrt{\log 2k} R_n
    + \sup_{w \in \cW} |\Bias(w)|.
  \end{align*}
  %
  Finally, we verify the variance bounds for the Gaussian process.
  With $\sigma^2(w)$ bounded above,
  %
  \begin{align*}
    \Var[G(w)]
    &=
    p(w)^\T H^{-1}
    \Var\left[ \sum_{i=1}^n p(W_i) \varepsilon_i \right]
    H^{-1} p(w) \\
    &=
    p(w)^\T H^{-1}
    \E\left[\sum_{i=1}^n p(W_i) p(W_i)^\T \sigma^2(W_i) \right]
    H^{-1} p(w) \\
    &\lesssim
    \|p(w)\|_2^2 \|H^{-1}\|_2^2
    \|H\|_2
    \lesssim
    k/n.
  \end{align*}
  %
  Similarly, since $\sigma^2(w)$ is bounded away from zero,
  %
  \begin{align*}
    \Var[G(w)]
    &\gtrsim
    \|p(w)\|_2^2 \|H^{-1}\|_2^2
    \|H^{-1}\|_2^{-1}
    \gtrsim
    k/n.
  \end{align*}

  \proofparagraph{bounding the bias}

  We delegate the task of carefully deriving bounds on the bias to
  \citet{cattaneo2020large}, who provide a high-level assumption on the
  approximation error in Assumption~4 and then use it to derive bias bounds in
  Section~3 of the form $\sup_{w \in \cW} |\Bias(w)| \lesssim_\P k^{-\gamma}$.
  This assumption is then verified for B-splines, wavelets and piecewise
  polynomials in their supplemental appendix.

\end{proof}

\begin{proof}[Proposition~\ref{pro:series_feasible}]
  \proofparagraph{infeasible supremum approximation}

  Provided that the bias is negligible,
  for all $s > 0$ we have
  %
  \begin{align*}
    &\sup_{t \in \R}
    \left|
    \P\left(
      \sup_{w \in \cW}
      \left|
      \frac{\hat\mu(w)-\mu(w)}{\sqrt{\rho(w,w)}}
      \right| \leq t
    \right)
    -
    \P\left(
      \sup_{w \in \cW}
      \left|
      \frac{G(w)}{\sqrt{\rho(w,w)}}
      \right| \leq t
    \right)
    \right| \\
    &\quad\leq
    \sup_{t \in \R}
    \P\left(
      t \leq
      \sup_{w \in \cW}
      \left|
      \frac{G(w)}{\sqrt{\rho(w,w)}}
      \right|
      \leq t + s
    \right)
    +
    \P\left(
      \sup_{w \in \cW}
      \left|
      \frac{\hat\mu(w)-\mu(w)-G(w)}{\sqrt{\rho(w,w)}}
      \right| > s
    \right).
  \end{align*}
  %
  By the Gaussian anti-concentration result given as Corollary~2.1 in
  \citet{chernozhukov2014anti} applied to a discretization of $\cW$, the first
  term is at most $s \sqrt{\log n}$ up to a constant factor, and the second
  term converges to zero whenever
  $\frac{1}{s} \left( \frac{k^3 (\log k)^3}{n} \right)^{1/6} \to 0$.
  Thus a suitable value of $s$ exists whenever $\frac{k^3(\log n)^6}{n} \to 0$.

  \proofparagraph{feasible supremum approximation}

  By \citet[Lemma~3.1]{chernozhukov2013gaussian} and discretization,
  with $\rho(w,w') = \E[\hat\rho(w,w')]$,
  %
  \begin{align*}
    &\sup_{t \in \R}
    \left|
    \P\left(
      \sup_{w \in \cW}
      \left|
      \frac{\hat G(w)}{\sqrt{\hat\rho(w,w)}}
      \right|
      \leq t \biggm| \bW, \bY
    \right)
    - \P\left(
      \left|
      \frac{G(w)}{\sqrt{\rho(w,w)}}
      \right|
      \leq t
    \right)
    \right| \\
    &\quad\lesssim_\P
    \sup_{w,w' \in \cW}
    \left|
    \frac{\hat\rho(w,w')}
    {\sqrt{\hat\rho(w,w)\hat\rho(w',w')}}
    - \frac{\rho(w,w')}
    {\sqrt{\rho(w,w)\rho(w',w')}}
    \right|^{1/3}
    (\log n)^{2/3} \\
    &\quad\lesssim_\P
    \left(\frac n k \right)^{1/3}
    \sup_{w,w' \in \cW} |\hat\rho(w,w') - \rho(w,w')|^{1/3}
    (\log n)^{2/3} \\
    &\quad\lesssim_\P
    \left( \frac{n (\log n)^2}{k} \right)^{1/3}
    \sup_{w,w' \in \cW}
    \left|
    p(w)^\T \hat H^{-1}
    \left(
      \hat{\Var}[S]
      - \Var[S]
    \right)
    \hat H^{-1} p(w')
    \right|^{1/3} \\
    &\quad\lesssim_\P
    \left( \frac{k (\log n)^2}{n} \right)^{1/3}
    \left\|
    \hat{\Var}[S]
    - \Var[S]
    \right\|_2^{1/3},
  \end{align*}
  %
  and goes to zero in probability whenever
  $\frac{k (\log n)^2}{n}
  \big\| \hat{\Var}[S] - \Var[S] \big\|_2 \to_\P 0$.
  For the plug-in estimator,
  %
  \begin{align*}
    &\left\|
    \hat{\Var}[S]
    - \Var[S]
    \right\|_2
    =
    \left\|
    \sum_{i=1}^n
    p(W_i) p(W_i^\T)
    \hat\sigma^2(W_i)
    - n \E\left[
      p(W_i) p(W_i^\T)
      \sigma^2(W_i)
    \right]
    \right\|_2 \\
    &\quad\lesssim_\P
    \sup_{w \in \cW}
    |\hat{\sigma}^2(w)-\sigma^2(w)|
    \, \big\| \hat H \big\|_2 \\
    &\qquad+
    \left\|
    \sum_{i=1}^n
    p(W_i) p(W_i^\T)
    \sigma^2(W_i)
    - n \E\left[
      p(W_i) p(W_i^\T)
      \sigma^2(W_i)
    \right]
    \right\|_2 \\
    &\quad\lesssim_\P
    \frac{n}{k}
    \sup_{w \in \cW}
    |\hat{\sigma}^2(w)-\sigma^2(w)|
    + \sqrt{n k},
  \end{align*}
  %
  where the second term is bounded by the same argument
  used to bound $\|\hat H - H\|_1$.
  Thus, the feasible approximation is valid whenever
  $(\log n)^2 \sup_{w \in \cW}
  |\hat{\sigma}^2(w)-\sigma^2(w)| \to_\P 0$
  and $\frac{k^3 (\log n)^4}{n} \to 0$.
  The validity of the uniform confidence band follows immediately.
  %
\end{proof}

\begin{proof}[Proposition~\ref{pro:local_poly}]

  We apply Proposition~\ref{pro:emp_proc}
  with the metric $d(f_w, f_{w'}) = \|w-w'\|_2$
  and the function class
  %
  \begin{align*}
    \cF
    &=
    \left\{
      (W_i, \varepsilon_i) \mapsto
      e_1^\T H(w)^{-1} K_h(W_i-w) p_h(W_i-w)
      \varepsilon_i
      :\ w \in \cW
    \right\},
  \end{align*}
  %
  with $\psi$ chosen as a suitable Bernstein Orlicz function.

  \proofparagraph{bounding $H(w)^{-1}$}

  Recall that
  $H(w) = \sum_{i=1}^n \E[K_h(W_i-w) p_h(W_i-w)p_h(W_i-w)^\T]$
  and let $a(w) \in \R^k$ with $\|a(w)\|_2 = 1$.
  Since the density of $W_i$ is bounded away from zero on $\cW$,
  %
  \begin{align*}
    a(w)^\T H(w) a(w)
    &=
    n \E\left[
      \big( a(w)^\T p_h(W_i-w) \big)^2
      K_h(W_i-w)
    \right] \\
    &\gtrsim
    n \int_\cW
    \big( a(w)^\T p_h(u-w) \big)^2
    K_h(u-w)
    \diff{u} \\
    &\gtrsim
    n \int_{\frac{\cW-w}{h}}
    \big( a(w)^\T p(u) \big)^2
    K(u)
    \diff{u}.
  \end{align*}
  %
  This is continuous in $a(w)$ on the compact set
  $\|a(w)\|_2 = 1$
  and $p(u)$ forms a polynomial basis so
  $a(w)^\T p(u)$ has finitely many zeroes.
  Since $K(u)$ is compactly supported
  and $h \to 0$,
  the above integral is eventually strictly positive
  for all $x \in \cW$,
  and hence is bounded below uniformly in $w \in \cW$
  by a positive constant.
  Therefore
  $\sup_{w \in \cW} \|H(w)^{-1}\|_2 \lesssim 1/n$.

  \proofparagraph{bounding $\beta_\delta$}

  Let $\cF_\delta$ be a $\delta$-cover of $(\cF, d)$
  with cardinality $|\cF_\delta| \asymp \delta^{-m}$
  and let
  $\cF_\delta(W_i, \varepsilon_i)
  = \big(f(W_i, \varepsilon_i) : f\in \cF_\delta\big)$.
  Define the truncated errors
  $\tilde\varepsilon_i =
  \varepsilon_i\I\{-a \log n \leq \varepsilon_i \leq b \log n\}$
  and note that
  $\E\big[e^{|\varepsilon_i|/C_\varepsilon}\big] < \infty$
  implies that
  $\P(\exists i: \tilde\varepsilon_i \neq \varepsilon_i)
  \lesssim n^{1-(a \vee b)/C_\varepsilon}$.
  Hence, by choosing $a$ and $b$ large enough,
  with high probability, we can replace all
  $\varepsilon_i$ by $\tilde\varepsilon_i$.
  Further, it is always possible to increase either $a$ or $b$
  along with some randomization to ensure that
  $\E[\tilde\varepsilon_i] = 0$.
  Since $K$ is bounded and compactly supported,
  $W_i$ has a bounded density and
  $|\tilde\varepsilon_i| \lesssim \log n$,
  %
  \begin{align*}
    \bigvvvert
    f(W_i, \tilde\varepsilon_i)
    \bigvvvert_2
    &=
    \E\left[
      \left|
      e_1^\T H(w)^{-1} K_h(W_i-w) p_h(W_i-w)
      \tilde\varepsilon_i
      \right|^2
    \right]^{1/2} \\
    &\leq
    \E\left[
      \|H(w)^{-1}\|_2^2
      K_h(W_i-w)^2
      \|p_h(W_i-w)\|_2^2
      \sigma^2(W_i)
    \right]^{1/2} \\
    &\lesssim
    n^{-1}
    \E\left[
      K_h(W_i-w)^2
    \right]^{1/2}
    \lesssim
    n^{-1}
    h^{-m / 2}, \\
    \bigvvvert
    f(W_i, \tilde\varepsilon_i)
    \bigvvvert_\infty
    &\leq
    \bigvvvert
    \|H(w)^{-1}\|_2
    K_h(W_i-w)
    \|p_h(W_i-w)\|_2
    |\tilde\varepsilon_i|
    \bigvvvert_\infty \\
    &\lesssim
    n^{-1}
    \bigvvvert
    K_h(W_i-w)
    \bigvvvert_\infty
    \log n
    \lesssim
    n^{-1}
    h^{-m}
    \log n.
  \end{align*}
  %
  Therefore
  %
  \begin{align*}
    \E\left[
      \|\cF_\delta(W_i, \tilde\varepsilon_i)\|_2^2
      \|\cF_\delta(W_i, \tilde\varepsilon_i)\|_\infty
    \right]
    &\leq
    \sum_{f\in\cF_\delta}
    \bigvvvert f(W_i, \tilde\varepsilon_i) \bigvvvert_2^2
    \, \max_{f\in\cF_\delta}
    \bigvvvert f(W_i, \tilde\varepsilon_i) \bigvvvert_\infty \\
    &\lesssim
    n^{-3} \delta^{-m} h^{-2m} \log n.
  \end{align*}
  %
  Let
  $V_i(\cF_\delta) =
  \E\big[\cF_\delta(W_i, \tilde\varepsilon_i)
    \cF_\delta(W_i, \tilde\varepsilon_i)^\T
  \mid \cH_{i-1}\big]$
  and $Z_i \sim \cN(0, I_d)$ be i.i.d.\ and
  independent of $\cH_n$.
  Note that
  $V_i(f,f) = \E[f(W_i, \tilde\varepsilon_i)^2 \mid W_i]
  \lesssim n^{-2} h^{-2m}$
  and
  $\E[V_i(f,f)] = \E[f(W_i, \tilde\varepsilon_i)^2]
  \lesssim n^{-2} h^{-m}$.
  Thus by Lemma~\ref{lem:gaussian_useful},
  %
  \begin{align*}
    \E\left[
      \big\| V_i(\cF_\delta)^{1/2} Z_i \big\|^2_2
      \big\| V_i(\cF_\delta)^{1/2} Z_i \big\|_\infty
    \right]
    &=
    \E\left[
      \E\left[
        \big\| V_i(\cF_\delta)^{1/2} Z_i \big\|^2_2
        \big\| V_i(\cF_\delta)^{1/2} Z_i \big\|_\infty
        \mid \cH_n
      \right]
    \right] \\
    &\leq
    4 \sqrt{\log 2|\cF_\delta|}
    \,\E\Bigg[
      \max_{f \in \cF_\delta} \sqrt{V_i(f,f)}
      \sum_{f \in \cF_\delta} V_i(f,f)
    \Bigg] \\
    &\lesssim
    n^{-3}
    h^{-2m}
    \delta^{-m}
    \sqrt{\log(1/\delta)}.
  \end{align*}
  %
  Thus since $\log(1/\delta) \asymp \log(1/h) \asymp\log n$,
  %
  \begin{align*}
    \beta_\delta
    &=
    \sum_{i=1}^n
    \E\left[
      \|\cF_\delta(W_i, \tilde\varepsilon_i)\|_2^2
      \|\cF_\delta(W_i, \tilde\varepsilon_i)\|_\infty
      + \big\| V_i(\cF_\delta)^{1/2} Z_i \big\|^2_2
      \big\| V_i(\cF_\delta)^{1/2} Z_i \big\|_\infty
    \right] \\
    &\lesssim
    \frac{\log n}
    {n^2 h^{2m} \delta^m}.
  \end{align*}

  \proofparagraph{bounding $\Omega_\delta$}

  Let $C_K>0$ be the radius of a $\ell^2$-ball
  containing the support of $K$
  and note that
  %
  \begin{align*}
    \left|
    V_i(f,f')
    \right|
    &=
    \Big|
    \E\Big[
      e_1^\T H(w)^{-1}
      p_h(W_i-w)
      e_1^\T H(w')^{-1}
      p_h(W_i-w') \\
      &\qquad\times
      K_h(W_i-w)
      K_h(W_i-w')
      \tilde\varepsilon_i^2
      \Bigm| \cH_{i-1}
    \Big]
    \Big| \\
    &\lesssim
    n^{-2}
    K_h(W_i-w)
    K_h(W_i-w') \\
    &\lesssim
    n^{-2}
    h^{-m}
    K_h(W_i-w)
    \I\{\|w-w'\|_2 \leq 2 C_K h\}.
  \end{align*}
  %
  Since $W_i$ are $\alpha$-mixing
  with $\alpha(j) < e^{-2j / C_\alpha}$,
  Lemma~\ref{lem:variance_mixing}\ref{eq:variance_mixing_exponential}
  with $r=3$ gives
  %
  \begin{align*}
    &\Var\left[
      \sum_{i=1}^n V_i(f,f')
    \right] \\
    &\quad\lesssim
    \sum_{i=1}^n
    \E\left[
      |V_i(f,f')|^3
    \right] ^{2/3}
    \lesssim
    n^{-3} h^{-2m}
    \E\left[
      K_h(W_i-w)^3
    \right] ^{2/3}
    \I\{\|w-w'\|_2 \leq 2 C_K h\} \\
    &\quad\lesssim
    n^{-3} h^{-2m}
    (h^{-2m})^{2/3}
    \I\{\|w-w'\|_2 \leq 2 C_K h\} \\
    &\quad\lesssim
    n^{-3} h^{-10m/3}
    \I\{\|w-w'\|_2 \leq 2 C_K h\}.
  \end{align*}
  %
  Therefore, by Jensen's inequality,
  %
  \begin{align*}
    \E\big[ \|\Omega_\delta\|_2 \big]
    &\leq
    \E\big[ \|\Omega_\delta\|_\rF \big]
    \leq
    \E\Bigg[
      \sum_{f,f' \in \cF_\delta}
      (\Omega_\delta)_{f,f'}^2
    \Bigg]^{1/2}
    \leq
    \Bigg(
      \sum_{f,f' \in \cF_\delta}
      \Var\left[
        \sum_{i=1}^n V_i(f,f')
      \right]
    \Bigg)^{1/2} \\
    &\lesssim
    n^{-3/2} h^{-5m/3}
    \Bigg(
      \sum_{f,f' \in \cF_\delta}
      \I\{\|w-w'\|_2 \leq 2 C_K h\}
    \Bigg)^{1/2} \\
    &\lesssim
    n^{-3/2} h^{-5m/3}
    \big(h^{m} \delta^{-2m} \big)^{1/2}
    \lesssim
    n^{-3/2}
    h^{-7m/6}
    \delta^{-m}.
  \end{align*}
  %
  Note that we could have used
  $\|\cdot\|_1$ rather than $\|\cdot\|_\rF$,
  but this term is negligible either way.

  \proofparagraph{regularity of the stochastic processes}

  For each $f, f' \in \cF$,
  define the mean-zero and $\alpha$-mixing random variables
  %
  \begin{align*}
    u_i(f,f')
    &=
    e_1^\T
    \big(
      H(w)^{-1} K_h(W_i-w) p_h(W_i-w)
      - H(w')^{-1} K_h(W_i-w') p_h(W_i-w')
    \big)
    \tilde\varepsilon_i.
  \end{align*}
  %
  Note that for all $1 \leq j \leq k$,
  by the Lipschitz property of the kernel and monomials,
  %
  \begin{align*}
    &\left|
    K_h(W_i-w) - K_h(W_i-w')
    \right| \\
    &\quad\lesssim
    h^{-m-1}
    \|w-w'\|_2
    \big(
      \I\{\|W_i-w\| \leq C_K h\}
      + \I\{\|W_i-w'\| \leq C_K h\}
    \big), \\
    &\left|
    p_h(W_i-w)_j - p_h(W_i-w')_j
    \right|
    \lesssim
    h^{-1}
    \|w-w'\|_2,
  \end{align*}
  %
  to deduce that for any $1 \leq j,l \leq k$,
  %
  \begin{align*}
    \big| H(w)_{j l} - H(w')_{j l} \big|
    &=
    \big|
    n \E\big[
      K_h(W_i-w) p_h(W_i-w)_j p_h(W_i-w)_l \\
      &\qquad-
      K_h(W_i-w') p_h(W_i-w')_j p_h(W_i-w')_l
    \big]
    \big| \\
    &\leq
    n\E\left[
      \left|
      K_h(W_i-w) - K_h(W_i-w')
      \right|
      \left|
      p_h(W_i-w)_j
      p_h(W_i-w)_l
      \right|
    \right] \\
    &\quad+
    n\E\left[
      \left|
      p_h(W_i-w)_j - p_h(W_i-w')_j
      \right|
      \left|
      K_h(W_i-w')
      p_h(W_i-w)_l
      \right|
    \right] \\
    &\quad+
    n\E\left[
      \left|
      p_h(W_i-w)_l - p_h(W_i-w')_l
      \right|
      \left|
      K_h(W_i-w')
      p_h(W_i-w')_j
      \right|
    \right] \\
    &\lesssim
    n h^{-1}\|w-w'\|_2.
  \end{align*}
  %
  Therefore as the dimension of the matrix $H(w)$ is fixed,
  %
  \begin{align*}
    \big\| H(w)^{-1} - H(w')^{-1} \big\|_2
    &\leq
    \big\| H(w)^{-1}\big\|_2
    \big\| H(w')^{-1}\big\|_2
    \big\| H(w) - H(w') \big\|_2
    \lesssim
    \frac{\|w-w'\|_2}{n h}.
  \end{align*}
  %
  Hence
  %
  \begin{align*}
    \big| u_i(f,f') \big|
    &\leq
    \big\|
    H(w)^{-1} K_h(W_i-w) p_h(W_i-w)
    - H(w')^{-1} K_h(W_i-w') p_h(W_i-w')
    \tilde\varepsilon_i
    \big\|_2 \\
    &\leq
    \big\| H(w)^{-1} - H(w')^{-1} \big\|_2
    \big\| K_h(W_i-w) p_h(W_i-w)
    \tilde\varepsilon_i
    \big\|_2 \\
    &\quad+
    \big| K_h(W_i-w) - K_h(W_i-w') \big|
    \big\| H(w')^{-1} p_h(W_i-w)
    \tilde\varepsilon_i
    \big\|_2 \\
    &\quad+
    \big\| p_h(W_i-w) - p_h(W_i-w') \big\|_2
    \big\| H(w')^{-1} K_h(W_i-w')
    \tilde\varepsilon_i \big\|_2 \\
    &\lesssim
    \frac{\|w-w'\|_2}{n h}
    \big| K_h(W_i-w) \tilde\varepsilon_i \big|
    + \frac{1}{n}
    \big| K_h(W_i-w) - K_h(W_i-w') \big|
    \,|\tilde\varepsilon_i| \\
    &\lesssim
    \frac{\|w-w'\|_2 \log n}{n h^{m+1}},
  \end{align*}
  %
  and from the penultimate line, we also deduce that
  %
  \begin{align*}
    \Var[u_i(f,f')]
    &\lesssim
    \frac{\|w-w'\|_2^2}{n^2h^2}
    \E\left[
      K_h(W_i-w)^2 \sigma^2(X_i)
    \right] \\
    &\quad+
    \frac{1}{n^2}
    \E\left[
      \big( K_h(W_i-w) - K_h(W_i-w') \big)^2
      \sigma^2(X_i)
    \right] \\
    &\lesssim
    \frac{\|w-w'\|_2^2}{n^2h^{m+2}}.
  \end{align*}
  %
  Further, $\E[u_i(f,f') u_j(f,f')] = 0$ for $i \neq j$ so
  by Lemma~\ref{lem:exponential_mixing}\ref{eq:exponential_mixing_bernstein},
  for a constant $C_1>0$,
  %
  \begin{align*}
    \P\left(
      \Big| \sum_{i=1}^n u_i(f,f') \Big|
      \geq \frac{C_1 \|w-w'\|_2}{\sqrt n h^{m/2+1}}
      \left(
        \sqrt{t}
        + \sqrt{\frac{(\log n)^2}{n h^m}} \sqrt t
        + \sqrt{\frac{(\log n)^6}{n h^m}} t
      \right)
    \right)
    &\leq
    C_1 e^{-t}.
  \end{align*}
  %
  Therefore, adjusting the constant if necessary
  and since $n h^{m} \gtrsim (\log n)^7$,
  %
  \begin{align*}
    \P\left(
      \Big| \sum_{i=1}^n u_i(f,f') \Big|
      \geq
      \frac{C_1 \|w-w'\|_2}{\sqrt{n} h^{m/2+1}}
      \left(
        \sqrt{t} + \frac{t}{\sqrt{\log n}}
      \right)
    \right)
    &\leq
    C_1 e^{-t}.
  \end{align*}
  %
  \citet[Lemma~2]{van2013bernstein} with
  $\psi(x) =
  \exp\Big(\big(\sqrt{1+2 x / \sqrt{\log n}}-1 \big)^2
  \log n \Big)-1$
  now shows that
  %
  \begin{align*}
    \Bigvvvert \sum_{i=1}^n u_i(f,f') \Bigvvvert_\psi
    &\lesssim
    \frac{\|w-w'\|_2}{\sqrt{n} h^{m/2+1}}
  \end{align*}
  %
  so we take $L = \frac{1}{\sqrt{n} h^{m/2+1}}$.
  Noting
  $\psi^{-1}(t) = \sqrt{\log(1+t)} + \frac{\log(1+t)}{2\sqrt{\log n}}$
  and $N_\delta \lesssim \delta^{-m}$,
  %
  \begin{align*}
    J_\psi(\delta)
    &=
    \int_0^\delta
    \psi^{-1}\big( N_\varepsilon \big)
    \diff{\varepsilon}
    + \delta
    \psi^{-1} \big( N_\delta \big)
    \lesssim
    \frac{\delta \log(1/\delta)}{\sqrt{\log n}}
    + \delta \sqrt{\log(1/\delta)}
    \lesssim
    \delta \sqrt{\log n}, \\
    J_2(\delta)
    &=
    \int_0^\delta
    \sqrt{\log N_\varepsilon}
    \diff{\varepsilon}
    \lesssim
    \delta \sqrt{\log(1/\delta)}
    \lesssim
    \delta \sqrt{\log n}.
  \end{align*}

  \proofparagraph{strong approximation}

  Recalling that
  $\tilde\varepsilon_i = \varepsilon_i$
  for all $i$ with high probability,
  by Proposition~\ref{pro:emp_proc},
  for all $t, \eta > 0$ there exists a
  zero-mean Gaussian process $T(w)$ satisfying
  \begin{align*}
    \E\left[
      \left(\sum_{i=1}^n f_w(W_i, \varepsilon_i)\right)
      \left(\sum_{i=1}^n f_{w'}(W_i, \varepsilon_i)\right)
    \right]
    &= \E\big[ T(w) T(w')
    \big]
  \end{align*}
  %
  for all $w, w' \in \cW$ and
  %
  \begin{align*}
    &\P\left(
      \sup_{w \in \cW}
      \left| \sum_{i=1}^n f_{w}(W_i, \varepsilon_i)
      - T(w) \right|
      \geq C_\psi(t + \eta)
    \right) \\
    &\quad\leq
    C_\psi
    \inf_{\delta > 0}
    \inf_{\cF_\delta}
    \Bigg\{
      \frac{\beta_\delta^{1/3} (\log 2 |\cF_\delta|)^{1/3}}{\eta }
      + \left(\frac{\sqrt{\log 2 |\cF_\delta|}
      \sqrt{\E\left[\|\Omega_\delta\|_2\right]}}{\eta }\right)^{2/3} \\
      &\qquad+
      \psi\left(\frac{t}{L J_\psi(\delta)}\right)^{-1}
      + \exp\left(\frac{-t^2}{L^2 J_2(\delta)^2}\right)
    \Bigg\} \\
    &\quad\leq
    C_\psi
    \Bigg\{
      \frac{
        \left(\frac{\log n} {n^2 h^{2m} \delta^{m}} \right)^{1/3}
      (\log n)^{1/3}}{\eta }
      + \left(\frac{\sqrt{\log n}
          \sqrt{n^{-3/2} h^{-7m/6} \delta^{-m}}
      }{\eta }\right)^{2/3} \\
      &\qquad+
      \psi\left(\frac{t}{\frac{1}{\sqrt{n} h^{m/2+1}}
      J_\psi(\delta)}\right)^{-1}
      + \exp\left(\frac{-t^2}{
          \left( \frac{1}{\sqrt{n} h^{m/2+1}} \right)^2
      J_2(\delta)^2}\right)
    \Bigg\} \\
    &\quad\leq
    C_\psi
    \Bigg\{
      \frac{
      (\log n)^{2/3}}{n^{2/3} h^{2m/3} \delta^{m/3} \eta}
      + \left(\frac{
        n^{-3/4} h^{-7m/12} \delta^{-m/2} \sqrt{\log n}}
      {\eta }\right)^{2/3} \\
      &\qquad+
      \psi\left(\frac{t\sqrt{n} h^{m/2+1}}
      {\delta \sqrt{\log n}}\right)^{-1}
      + \exp\left(\frac{-t^2n h^{m+2}}
      {\delta^2 \log n}\right)
    \Bigg\}.
  \end{align*}
  %
  Noting $\psi(x) \geq e^{x^2/4}$ for $x \leq 4 \sqrt{\log n}$,
  any $R_n \to \infty$ gives the probability bound
  %
  \begin{align*}
    \sup_{w \in \cW}
    \left| \sum_{i=1}^n f_{w}(W_i, \varepsilon_i)
    - T(w) \right|
    &\lesssim_\P
    \frac{(\log n)^{2/3}}{n^{2/3} h^{2m/3} \delta^{m/3}} R_n
    + \frac{\sqrt{\log n}}{n^{3/4} h^{7m/12} \delta^{m/2}} R_n
    + \frac{\delta \sqrt{\log n}} {\sqrt{n} h^{m/2+1}}.
  \end{align*}
  %
  Optimizing over $\delta$ gives
  $\delta \asymp \left(\frac{\log n}{n h^{m-6}}\right)^{\frac{1}{2m+6}}
  = h \left( \frac{\log n}{n h^{3m}} \right)^{\frac{1}{2m+6}}$
  and so
  %
  \begin{align*}
    \sup_{w \in \cW}
    \left| \sum_{i=1}^n f_{w}(W_i, \varepsilon_i)
    - T(w) \right|
    &\lesssim_\P
    \left(
      \frac{(\log n)^{m+4}}{n^{m+4}h^{m(m+6)}}
    \right)^{\frac{1}{2m+6}} R_n.
  \end{align*}

  \proofparagraph{convergence of $\hat H(w)$}

  For $1 \leq j,l \leq k$
  define the zero-mean random variables
  %
  \begin{align*}
    u_{i j l}(w)
    &=
    K_h(W_i-w) p_h(W_i-w)_j p_h(W_i-w)_l \\
    &\quad-
    \E\big[K_h(W_i-w) p_h(W_i-w)_j p_h(W_i-w)_l \big]
  \end{align*}
  %
  and note that
  $|u_{i j l}(w)| \lesssim h^{-m}$.
  By Lemma~\ref{lem:exponential_mixing}\ref{eq:exponential_mixing_bounded}
  for a constant $C_2 > 0$ and all $t > 0$,
  %
  \begin{align*}
    \P\left(
      \left|
      \sum_{i=1}^n
      u_{i j l}(w)
      \right|
      > C_2 h^{-m} \big( \sqrt{n t}
      + (\log n)(\log \log n) t \big)
    \right)
    &\leq
    C_2 e^{-t}.
  \end{align*}
  %
  Further, note that by Lipschitz properties,
  %
  \begin{align*}
    \left|
    \sum_{i=1}^n u_{i j l}(w)
    - \sum_{i=1}^n  u_{i j l}(w')
    \right|
    &\lesssim
    h^{-m-1} \|w-w'\|_2
  \end{align*}
  %
  so there is a $\delta$-cover of $(\cW, \|\cdot\|_2)$
  with size at most $n^a \delta^{-a}$ for some $a > 0$.
  Adjusting $C_2$,
  %
  \begin{align*}
    \P\left(
      \sup_{w \in \cW}
      \left|
      \sum_{i=1}^n
      u_{i j l}(w)
      \right|
      > C_2 h^{-m} \big( \sqrt{n t}
      + (\log n)(\log \log n) t \big)
      + C_2 h^{-m-1} \delta
    \right)
    &\leq
    C_2 n^a \delta^{-a}
    e^{-t}
  \end{align*}
  %
  and hence
  %
  \begin{align*}
    \sup_{w \in \cW}
    \left|
    \sum_{i=1}^n
    u_{i j l}(w)
    \right|
    &\lesssim_\P
    h^{-m} \sqrt{n \log n}
    + h^{-m} (\log n)^3
    \lesssim_\P
    \sqrt{\frac{n \log n}{h^{2m}}}.
  \end{align*}
  %
  Therefore
  %
  \begin{align*}
    \sup_{w\in\cW} \|\hat H(w)-H(w)\|_2
    &\lesssim_\P
    \sqrt{\frac{n \log n}{h^{2m}}}.
  \end{align*}

  \proofparagraph{bounding the matrix term}

  Firstly note that,
  since $\sqrt{\frac{\log n}{n h^{2m}}} \to 0$,
  we have that uniformly in $w \in \cW$
  %
  \begin{align*}
    \|\hat H(w)^{-1}\|_2
    \leq
    \frac{\|H(w)^{-1}\|_2}
    {1 - \|\hat H(w)-H(w)\|_2 \|H(w)^{-1}\|_2}
    &\lesssim_\P
    \frac{1/n}
    {1 - \sqrt{\frac{n \log n}{h^{2m}}} \frac{1}{n}}
    \lesssim_\P
    \frac{1}{n}.
  \end{align*}
  %
  Therefore
  %
  \begin{align*}
    &\sup_{w \in \cW}
    \big|
    e_1^\T \big(\hat H(w)^{-1} - H(w)^{-1}\big)
    S(w)
    \big|
    \leq
    \sup_{w \in \cW}
    \big\|\hat H(w)^{-1} - H(w)^{-1}\big\|_2
    \|S(w)\|_2 \\
    &\quad\leq
    \sup_{w \in \cW}
    \big\|\hat H(w)^{-1}\big\|_2
    \big\|H(w)^{-1}\big\|_2
    \big\|\hat H(w) - H(w)\big\|_2
    \|S(w)\|_2 \\
    &\quad\lesssim_\P
    \sqrt{\frac{\log n}{n^3 h^{2m}}}
    \sup_{w \in \cW}
    \|S(w)\|_2.
  \end{align*}
  %
  Now for $1 \leq j \leq k$ write
  $u_{i j}(w) = K_h(W_i-w) p_h(W_i-w)_j \tilde \varepsilon_i$
  so that $S(w)_j = \sum_{i=1}^n u_{i j}(w)$ with high probability.
  Note that $u_{i j}(w)$ are zero-mean with
  $\Cov[u_{i j}(w), u_{i' j}(w)] = 0$ for $ i \neq i'$.
  Also $|u_{i j}(w)| \lesssim h^{-m} \log n$
  and $\Var[u_{i j}(w)] \lesssim h^{-m}$.
  By Lemma~\ref{lem:exponential_mixing}\ref{eq:exponential_mixing_bernstein}
  for a constant $C_3>0$,
  %
  \begin{align*}
    \P\left(
      \Big| \sum_{i=1}^n u_{i j}(w) \Big|
      \geq C_3 \big( (h^{-m/2} \sqrt n + h^{-m} \log n) \sqrt t
      + h^{-m} (\log n)^3 t \big)
    \right)
    &\leq
    C_3 e^{-t}, \\
    \P\left(
      \Big| \sum_{i=1}^n u_{i j}(w) \Big|
      >
      C_3 \left(
        \sqrt{\frac{tn}{h^{m}}}
        + \frac{t(\log n)^3}{h^{m}}
      \right)
    \right)
    &\leq
    C_3 e^{-t},
  \end{align*}
  %
  where we used $n h^{m} \gtrsim (\log n)^2$
  and adjusted the constant if necessary.
  As before,
  $u_{i j}(w)$ is Lipschitz in $w$ with a constant which is at most
  polynomial in $n$,
  so for some $a>0$
  %
  \begin{align*}
    \P\left(
      \sup_{w \in \cW}
      \Big| \sum_{i=1}^n u_{i j}(w) \Big|
      >
      C_3 \left(
        \sqrt{\frac{tn}{h^{m}}}
        + \frac{t(\log n)^3}{h^{m}}
      \right)
    \right)
    &\leq
    C_3 n^a e^{-t}, \\
    \sup_{w \in \cW}
    \|S(w)\|_2
    \lesssim_\P
    \sqrt{\frac{n \log n}{h^{m}}}
    + \frac{(\log n)^4}{h^{m}}
    &\lesssim_\P
    \sqrt{\frac{n \log n}{h^{m}}}
  \end{align*}
  %
  as $n h^m \gtrsim (\log n)^7$.
  Finally
  %
  \begin{align*}
    \sup_{w \in \cW}
    \big|
    e_1^\T \big(\hat H(w)^{-1} - H(w)^{-1}\big)
    S(w)
    \big|
    &\lesssim_\P
    \sqrt{\frac{\log n}{n^3 h^{2m}}}
    \sqrt{\frac{n \log n}{h^{m}}}
    \lesssim_\P
    \frac{\log n}{\sqrt{n^2 h^{3m}}}.
  \end{align*}

  \proofparagraph{bounding the bias}

  Since $\mu \in \cC^\gamma$, we have, by the multivariate version of Taylor's
  theorem,
  %
  \begin{align*}
    \mu(W_i)
    &=
    \sum_{|\kappa|=0}^{\gamma-1}
    \frac{1}{\kappa!}
    \partial^{\kappa} \mu(w)
    (W_i-w)^\kappa
    + \sum_{|\kappa|=\gamma}
    \frac{1}{\kappa!}
    \partial^{\kappa} \mu(w')
    (W_i-w)^\kappa
  \end{align*}
  %
  for some $w'$ on the line segment connecting
  $w$ and $W_i$.
  Now since $p_h(W_i-w)_1 = 1$,
  %
  \begin{align*}
    &e_1^\T \hat H(w)^{-1}
    \sum_{i=1}^n K_h(W_i-w) p_h(W_i-w) \mu(w) \\
    &\quad=
    e_1^\T \hat H(w)^{-1}
    \sum_{i=1}^n K_h(W_i-w) p_h(W_i-w) p_h(W_i-w)^\T e_1 \mu(w)
    = e_1^\T e_1 \mu(w) = \mu(w).
  \end{align*}
  %
  Therefore
  %
  \begin{align*}
    \Bias(w)
    &=
    e_1^\T \hat H(w)^{-1}
    \sum_{i=1}^n K_h(W_i-w) p_h(W_i-w) \mu(W_i)
    - \mu(w) \\
    &=
    e_1^\T \hat H(w)^{-1}
    \sum_{i=1}^n K_h(W_i-w) p_h(W_i-w) \\
    &\quad\times
    \Bigg(
      \sum_{|\kappa|=0}^{\gamma-1}
      \frac{1}{\kappa!}
      \partial^{\kappa} \mu(w)
      (W_i-w)^\kappa
      + \sum_{|\kappa|=\gamma}
      \frac{1}{\kappa!}
      \partial^{\kappa} \mu(w')
      (W_i-w)^\kappa
      - \mu(w)
    \Bigg) \\
    &=
    \sum_{|\kappa|=1}^{\gamma-1}
    \frac{1}{\kappa!}
    \partial^{\kappa} \mu(w)
    e_1^\T \hat H(w)^{-1}
    \sum_{i=1}^n K_h(W_i-w) p_h(W_i-w)
    (W_i-w)^\kappa \\
    &\quad+
    \sum_{|\kappa|=\gamma}
    \frac{1}{\kappa!}
    \partial^{\kappa} \mu(w')
    e_1^\T \hat H(w)^{-1}
    \sum_{i=1}^n K_h(W_i-w) p_h(W_i-w)
    (W_i-w)^\kappa \\
    &=
    \sum_{|\kappa|=\gamma}
    \frac{1}{\kappa!}
    \partial^{\kappa} \mu(w')
    e_1^\T \hat H(w)^{-1}
    \sum_{i=1}^n K_h(W_i-w) p_h(W_i-w)
    (W_i-w)^\kappa,
  \end{align*}
  %
  where we used that
  $p_h(W_i-w)$ is a vector containing monomials
  in $W_i-w$ of order up to $\gamma$, so
  $e_1^\T \hat H(w)^{-1}
  \sum_{i=1}^n K_h(W_i-w) p_h(W_i-w)
  (W_i-w)^\kappa = 0$
  whenever $1 \leq |\kappa| \leq \gamma$.
  Finally
  %
  \begin{align*}
    &\sup_{w\in\cW}
    |\Bias(w)| \\
    &\quad=
    \sup_{w\in\cW}
    \left|
    \sum_{|\kappa|=\gamma}
    \frac{1}{\kappa!}
    \partial^{\kappa} \mu(w')
    e_1^\T \hat H(w)^{-1}
    \sum_{i=1}^n K_h(W_i-w) p_h(W_i-w)
    (W_i-w)^\kappa
    \right| \\
    &\quad\lesssim_\P
    \sup_{w\in\cW}
    \max_{|\kappa| = \gamma}
    \left|
    \partial^{\kappa} \mu(w')
    \right|
    \|\hat H(w)^{-1}\|_2
    \left\|
    \sum_{i=1}^n K_h(W_i-w) p_h(W_i-w)
    \right\|_2
    h^\gamma \\
    &\quad\lesssim_\P
    \frac{h^\gamma}{n}
    \sup_{w\in\cW}
    \left\|
    \sum_{i=1}^n K_h(W_i-w) p_h(W_i-w)
    \right\|_2.
  \end{align*}
  %
  Now write
  $\tilde u_{i j}(w) = K_h(W_i-w)p_h(W_i-w)_j$
  and note that
  $|\tilde u_{i j}(w)| \lesssim h^{-m}$
  and
  $\E[\tilde u_{i j}(w)] \lesssim 1$.
  By Lemma~\ref{lem:exponential_mixing}%
  \ref{eq:exponential_mixing_bounded},
  for a constant $C_4$,
  %
  \begin{align*}
    \P\left(
      \left|
      \sum_{i=1}^n \tilde u_{i j}(w)
      - \E\left[
        \sum_{i=1}^n \tilde u_{i j}(w)
      \right]
      \right|
      > C_4 h^{-m} \big( \sqrt{n t}
      + (\log n)(\log \log n) t \big)
    \right)
    &\leq
    C_4 e^{-t}.
  \end{align*}
  %
  As in previous parts, by Lipschitz properties,
  this implies
  %
  \begin{align*}
    \sup_{w \in \cW}
    \left|
    \sum_{i=1}^n \tilde u_{i j}(w)
    \right|
    &\lesssim_\P
    n
    \left(
      1 + \sqrt{\frac{\log n}{n h^{2m}}}
    \right)
    \lesssim_\P
    n.
  \end{align*}
  %
  Therefore
  $\sup_{w\in\cW} |\Bias(w)|
  \lesssim_\P n h^\gamma / n
  \lesssim_\P h^\gamma$.

  \proofparagraph{conclusion}

  By the previous parts,
  %
  \begin{align*}
    \sup_{w \in \cW}
    \left|\hat \mu(w) - \mu(w) - T(w) \right|
    &\leq
    \sup_{w \in \cW}
    \left|e_1^\T H(w)^{-1} S(w) - T(w) \right| \\
    &\quad+
    \sup_{w \in \cW}
    \left| e_1^\T \big(\hat H(w)^{-1} - H(w)^{-1}\big) S(w) \right|
    + \sup_{w \in \cW}
    |\Bias(w)| \\
    &\lesssim_\P
    \left(
      \frac{(\log n)^{m+4}}{n^{m+4}h^{m(m+6)}}
    \right)^{\frac{1}{2m+6}} R_n
    + \frac{\log n}{\sqrt{n^2 h^{3m}}}
    + h^\gamma \\
    &\lesssim_\P
    \frac{R_n}{\sqrt{n h^m}}
    \left(
      \frac{(\log n)^{m+4}}{n h^{3m}}
    \right)^{\frac{1}{2m+6}}
    + h^\gamma,
  \end{align*}
  %
  where the last inequality follows because
  $n h^{3m} \to \infty$
  and $\frac{1}{2m+6} \leq \frac{1}{2}$.
  Finally, we verify the upper and lower bounds
  on the variance of the Gaussian process.
  Since the spectrum of $H(w)^{-1}$
  is bounded above and below by $1/n$,
  %
  \begin{align*}
    \Var[T(w)]
    &=
    \Var\left[
      e_1^\T H(w)^{-1}
      \sum_{i=1}^n K_h(W_i-w) p_h(W_i-w) \varepsilon_i
    \right] \\
    &=
    e_1^\T H(w)^{-1}
    \Var\left[
      \sum_{i=1}^n K_h(W_i-w) p_h(W_i-w) \varepsilon_i
    \right]
    H(w)^{-1} e_1^\T \\
    &\lesssim
    \|H(w)^{-1}\|_2^2
    \max_{1 \leq j \leq k}
    \sum_{i=1}^n
    \Var\big[
      K_h(W_i-w) p_h(W_i-w)_j \sigma(W_i)
    \big] \\
    &\lesssim
    \frac{1}{n^2} n
    \frac{1}{h^m}
    \lesssim
    \frac{1}{n h^m}.
  \end{align*}
  %
  Similarly
  $\Var[T(w)] \gtrsim \frac{1}{n h^m}$
  by the same argument given to bound the eigenvalues of
  $H(w)^{-1}$.
  %
\end{proof}

\section{High-dimensional central limit theorems for martingales}%
\label{sec:high_dim_clt}

We present an application of our main results to
high-dimensional central limit theorems for martingales. Our main
contribution here is the generality of our results, which are broadly
applicable to martingale data and impose minimal extra assumptions. In exchange
for the scope and breadth of our results, we naturally do not necessarily
achieve state-of-the-art distributional approximation errors in certain special
cases, such as with independent data or when restricting the class of sets over
which the central limit theorem must hold. Extensions of our high-dimensional
central limit theorem results to mixingales and other approximate martingales,
along with third-order refinements and Gaussian mixture target distributions,
are possible through methods akin to those used to establish our main results
in Section~\ref{sec:main_results}, but we omit these for succinctness.

Our approach to deriving a high-dimensional martingale central limit theorem
proceeds as follows. Firstly, the upcoming Proposition~\ref{pro:clt} uses our
main result on martingale coupling (Corollary~\ref{cor:sa_martingale}) to
reduce the problem to that of providing anti-concentration results for
high-dimensional Gaussian vectors. We then demonstrate the utility of this
reduction by employing a few such anti-concentration methods from the existing
literature. Proposition~\ref{pro:bootstrap} gives a feasible implementation via
the Gaussian multiplier bootstrap, enabling valid
resampling-based inference using
the resulting conditional Gaussian distribution. Finally in
Section~\ref{sec:lp} we provide an example application: distributional
approximation for $\ell^p$-norms of high-dimensional martingale vectors
in Kolmogorov--Smirnov distance, relying on some recent results
concerning Gaussian perimetric inequalities
\citep{nazarov2003maximal,kozbur2021dimension,
giessing2023anti,chernozhukov2017detailed}.

We begin this section with some notation. Assume the setup of
Corollary~\ref{cor:sa_martingale} and suppose $\Sigma$ is
non-random. Let $\cA$ be a class of measurable subsets of
$\R^d$ and take $T \sim \cN(0, \Sigma)$.
For $\eta>0$ and $p \in [1, \infty]$ define the Gaussian perimetric quantity
%
\begin{align*}
  \Delta_p(\cA, \eta)
  &=
  \sup_{A\in \cA}
  \big\{\P(T\in A_p^\eta\setminus A)
  \vee \P(T\in A \setminus A_p^{-\eta})\big\},
\end{align*}
%
where $A_p^\eta = \{x \in \R^d : \|x - A\|_p \leq \eta\}$,
$A_p^{-\eta} = \R^d \setminus (\R^d \setminus A)_p^\eta$
and $\|x - A\|_p = \inf_{x' \in A} \|x - x'\|_p$.
Using this perimetric term allows us to convert coupling results
to central limit theorems as follows.
Denote by $\Gamma_p(\eta)$ the rate of strong approximation attained in
Corollary~\ref{cor:sa_martingale}:
%
\begin{align*}
  \Gamma_p(\eta)
  &=
  24 \left(
    \frac{\beta_{p,2} \phi_p(d)^2}{\eta^3}
  \right)^{1/3}
  + 17 \left(
    \frac{\E \left[ \|\Omega\|_2 \right] \phi_p(d)^2}{\eta^2}
  \right)^{1/3}.
\end{align*}

\begin{proposition}[High-dimensional central limit theorem for martingales]%
  \label{pro:clt}

  Assume the setup of Corollary~\ref{cor:sa_martingale},
  and take $\Sigma$ non-random.
  For a class $\cA$ of measurable subsets of $\R^d$,
  %
  \begin{equation}%
    \label{eq:high_dim_clt}
    \sup_{A\in \cA}
    \big|\P(S\in A) -\P(T\in A)\big|
    \leq \inf_{p \in [1, \infty]} \inf_{\eta>0}
    \big\{\Gamma_p(\eta) + \Delta_p(\cA, \eta) \big\}.
  \end{equation}
\end{proposition}

\begin{proof}[Proposition~\ref{pro:clt}]

  This follows from Strassen's theorem (Lemma~\ref{lem:strassen}), but we
  provide a proof for completeness.
  %
  \begin{align*}
    \P(S \in A)
    &\leq
    \P(T \in A)
    + \P(T \in A_p^\eta \setminus A)
    + \P(\|S - T\| > \eta)
  \end{align*}
  %
  and applying this to $\R^d \setminus A$ gives
  %
  \begin{align*}
    \P(S\in A)
    &=
    1 - \P(S\in \R^d \setminus A) \\
    &\geq
    1 - \P(T \in \R^d \setminus A)
    - \P(T \in (\R^d \setminus A)_p^\eta \setminus (\R^d \setminus A))
    - \P(\|S - T\| > \eta) \\
    &=
    \P(T \in A)
    - \P(T \in A \setminus A_p^{-\eta})
    - \P(\|S - T\| > \eta).
  \end{align*}
  %
  Since this holds for all $p \in [1, \infty]$,
  %
  \begin{align*}
    \sup_{A\in \cA}
    \big|\P(S\in A) -\P(T\in A)\big|
    &\leq
    \sup_{A \in \cA}
    \big\{\P(T \in A_p^\eta\setminus A)
    \vee \P(T \in A \setminus A_p^{-\eta})\big\} \\
    &\quad+
    \P(\|S - T\| > \eta) \\
    &\leq
    \inf_{p \in [1, \infty]} \inf_{\eta>0}
    \big\{\Gamma_p(\eta) + \Delta_p(\cA, \eta) \big\}.
  \end{align*}
  %
\end{proof}

The term $\Delta_p(\cA, \eta)$
in \eqref{eq:high_dim_clt} is a Gaussian anti-concentration quantity
so it depends on the law of $S$ only through the covariance matrix $\Sigma$.
A few results are available in the literature
for bounding this term.
For instance, with
$\cA = \cC = \{A \subseteq \R^d \text{ is convex}\}$,
\citet{nazarov2003maximal} showed
%
\begin{equation}%
  \label{eq:convex_anticonc}
  \Delta_2(\cC, \eta)
  \asymp
  \eta\sqrt{\|\Sigma^{-1}\|_{\rF}},
\end{equation}
%
whenever $\Sigma$ is invertible.
Then Proposition~\ref{pro:clt} with $p=2$
combined with \eqref{eq:convex_anticonc} yields for convex sets
%
\begin{align*}
  \sup_{A\in \cC}
  \big|\P(S\in A) -\P(T\in A)\big|
  &\lesssim
  \inf_{\eta > 0}
  \left\{
    \left(\frac{\beta_{p,2} d}{\eta^3}\right)^{1/3}
    + \left(\frac{\E[\|\Omega \|_2] d}{\eta^2}\right)^{1/3}
    + \eta \sqrt{\|\Sigma^{-1}\|_\rF}
  \right\}.
\end{align*}

Alternatively, one can take $\cA = \cR$,
the class of axis-aligned rectangles in $\R^d$.
By Nazarov's Gaussian perimetric inequality
\citep{nazarov2003maximal,chernozhukov2017central},
%
\begin{align}%
  \label{eq:rect_anticonc}
  \Delta_\infty(\cR, \eta)
  \leq \frac{\eta (\sqrt{2\log d} + 2)}{\sigma_{\min}}
\end{align}
%
whenever $\min_j \, \Sigma_{j j} \geq \sigma_{\min}^2$
for some $\sigma_{\min}>0$.
Proposition~\ref{pro:clt} with $p = \infty$
and \eqref{eq:rect_anticonc} yields
%
\begin{align*}%
  &\sup_{A\in \cR}
  \big|\P(S\in A) -\P(T\in A)\big| \\
  &\quad\lesssim
  \inf_{\eta > 0}
  \left\{
    \left(\frac{\beta_{\infty,2} \log 2d}{\eta^3}\right)^{1/3}
    + \left(\frac{\E[\|\Omega \|_2] \log 2d}{\eta^2}\right)^{1/3}
    + \frac{\eta \sqrt{\log 2d}}{\sigma_{\min}}
  \right\}.
\end{align*}
%
In situations where
$\liminf_n \min_j \, \Sigma_{j j} = 0$,
it may be possible in certain cases to regularize
the minimum variance away from zero and then apply
a Gaussian--Gaussian rectangular approximation result
such as Lemma~2.1 from \citet{chernozhukov2023nearly}.

\begin{remark}[Comparisons with the literature]

  The literature on high-dimensional central limit theorems
  has developed rapidly in recent years
  \citep[see][and references therein]{%
    zhai2018high,%
    koike2021notes,%
    buzun2022strong,%
    lopes2022central,%
    chernozhukov2023nearly%
  },
  particularly for the special case of
  sums of independent random vectors
  on the rectangular sets $\cR$.
  %For example, if $X_i$ are i.i.d.\ and bounded a.s.\
  %and if $\Var[X_i]$ has minimum eigenvalue bounded away from zero,
  %\citet[Theorem~2.1]{chernozhukov2023nearly} showed
  %
  %\begin{align*}
  %\sup_{A\in \cR}
  %\big|\P(S\in A) -\P(T\in A)\big|
  %&\lesssim
  %\frac{(\log d)^{3/2} \log n}{\sqrt n}.
  %\end{align*}
  %
  %For comparison, under the same conditions, we have
  %$\Omega=0$ by independence,
  %$\beta_{\infty,2} \lesssim n d \sqrt{\log d}$
  %and $\sigma_{\min} \gtrsim \sqrt n$;
  %our closest comparable result
  %\eqref{eq:clt_rectangle} yields
  %
  %\begin{align*}
  %\sup_{A\in \cR}
  %\big|\P(S\in A) -\P(T\in A)\big|
  %&\lesssim
  %\left(\frac{d(\log d)^3}{\sqrt{n}}\right)^{1/6}.
  %\end{align*}
  %
  Our corresponding results are rather weaker in terms of
  dependence on the dimension than for example
  \citet[Theorem~2.1]{chernozhukov2023nearly}.
  This is an inherent issue due to our approach
  of first
  %constructing a coupling for the high-dimensional
  %vector $S$ via Yurinskii's method,
  considering the class of all Borel sets
  and only afterwards specializing to the smaller class $\cR$,
  where sharper results in the literature directly target the
  Kolmogorov--Smirnov distance via Stein's method and Slepian interpolation.
  %The main contribution of this section is therefore to obtain
  %Gaussian distributional approximations
  %for high-dimensional martingale vectors,
  %a setting in which alternative central limit theorem proof strategies are not
  %readily available.
  %It may be possible to improve our approach
  %somewhat by adjusting
  %our proof strategy. If the family of sets under consideration
  %is substantially smaller than the family of Borel sets,
  %one might be able to
  %with a superior smoothing argument on $\cR$
  %skip Strassen's theorem entirely.
\end{remark}

Next, we present a version of Proposition~\ref{pro:clt} in which the covariance
matrix $\Sigma$ is replaced by an estimator $\hat \Sigma$. This ensures that
the associated conditionally Gaussian vector is feasible and can be resampled,
allowing Monte Carlo quantile estimation via a Gaussian
multiplier bootstrap.

\begin{proposition}[Bootstrap central limit theorem for martingales]%
  \label{pro:bootstrap}

  Assume the setup of Corollary~\ref{cor:sa_martingale},
  with $\Sigma$ non-random,
  and let $\hat \Sigma$ be an $\bX$-measurable random
  $d \times d$ positive semi-definite matrix,
  where $\bX = (X_1, \ldots, X_n)$.
  For a class $\cA$ of measurable subsets of $\R^d$,
  %
  \begin{align*}
    &\sup_{A\in \cA}
    \left|
    \P\big(S \in A\big)
    - \P\big(\hat \Sigma^{1/2} Z \in A \bigm| \bX \big)
    \right| \\
    &\quad\leq
    \inf_{p \in [1,\infty]} \inf_{\eta>0}
    \left\{ \Gamma_p(\eta) + 2 \Delta_p(\cA, \eta)
      + 2d \exp\left(\frac{-\eta^2}
        {2d^{2/p}\big\|\hat \Sigma^{1/2} - \Sigma^{1/2}\big\|_2^2}
      \right)
    \right\},
  \end{align*}
  %
  where $Z \sim \cN(0,I_d)$ is independent of $\bX$.
\end{proposition}

\begin{proof}[Proposition~\ref{pro:bootstrap}]

  Since $T = \Sigma^{1/2} Z$ is independent of $\bX$,
  %
  \begin{align*}
    &\left|
    \P\big(S \in A\big)
    - \P\left(\hat \Sigma^{1/2} Z \in A \bigm| \bX\right)
    \right| \\
    &\quad\leq
    \left|
    \P\big(S \in A\big)
    - \P\big(T \in A\big)
    \right|
    +\left|
    \P\big(\Sigma^{1/2} Z \in A\big)
    - \P\left(\hat \Sigma^{1/2} Z \in A \bigm| \bX\right)
    \right|.
  \end{align*}
  %
  The first term is bounded by Proposition~\ref{pro:clt};
  the second by Lemma~\ref{lem:feasible_gaussian}
  conditional on $\bX$.
  %
  \begin{align*}
    &\left|
    \P\big(S \in A\big)
    - \P\left(\hat \Sigma^{1/2} Z \in A \bigm| \bX\right)
    \right| \\
    &\quad\leq
    \Gamma_p(\eta) + \Delta_p(\cA, \eta)
    + \Delta_{p'}(\cA, \eta')
    + 2 d \exp \left( \frac{-\eta'^2}
      {2 d^{2/p'} \big\|\hat\Sigma^{1/2} - \Sigma^{1/2}\big\|_2^2}
    \right)
  \end{align*}
  %
  for all $A \in \cA$
  and any $p, p' \in [1, \infty]$ and $\eta, \eta' > 0$.
  Taking a supremum over $A$ and infima over
  $p = p'$ and $\eta = \eta'$ yields the result.
  Note that we need not insist that
  $p = p'$ and $\eta = \eta'$ in general.
  %
\end{proof}

A natural choice for $\hat\Sigma$ in certain situations is the sample
covariance matrix $\sum_{i=1}^n X_i X_i^\T$, or a correlation-corrected variant
thereof. In general, whenever $\hat \Sigma$ does not depend on unknown
quantities, one can sample from the law of $\hat T = \hat\Sigma^{1/2} Z$
conditional on $\bX$ to approximate the distribution of $S$.
Proposition~\ref{pro:bootstrap} verifies that this Gaussian multiplier
bootstrap approach is valid whenever $\hat\Sigma$ and $\Sigma$ are sufficiently
close. To this end, Theorem~X.1.1 in \citet{bhatia1997matrix} gives
$\big\|\hat\Sigma^{1/2} - \Sigma^{1/2}\big\|_2
\leq \big\|\hat\Sigma - \Sigma\big\|_2^{1/2}$
and Problem~X.5.5 in the same gives
$\big\|\hat\Sigma^{1/2} - \Sigma^{1/2}\big\|_2
\leq \big\|\Sigma^{-1/2}\big\|_2 \big\|\hat\Sigma - \Sigma\big\|_2$
when $\Sigma$ is invertible. The latter often gives a tighter bound when the
minimum eigenvalue of $\Sigma$ can be bounded away from zero, and consistency
of $\hat \Sigma$ can be established using a range of matrix concentration
inequalities.

In Section~\ref{sec:lp} we apply Proposition~\ref{pro:clt} to the special case
of approximating the distribution of the $\ell^p$-norm of a high-dimensional
martingale. Proposition~\ref{pro:bootstrap} is then used to ensure that
feasible distributional approximations are also available.

\subsection{Application: distributional approximation of martingale
\texorpdfstring{$\ell^p$}{lp}-norms}
\label{sec:lp}

In some empirical applications,
including nonparametric significance tests
\citep{lopes2020bootstrapping}
and nearest neighbor search procedures
\citep{biau2015high},
an estimator or test statistic
can be expressed under the null hypothesis
as the $\ell^p$-norm of a zero-mean
(possibly high-dimensional) martingale for some $p \in [1, \infty]$.
In the notation of Corollary~\ref{cor:sa_martingale},
it is therefore of interest to bound Kolmogorov--Smirnov
quantities of the form
%
\begin{align*}
  \sup_{t \geq 0}
  \big| \P( \|S\|_p \leq t)
  - \P( \|T\|_p \leq t) \big|.
\end{align*}
%
Let $\cB_p$ be the class of closed $\ell^p$-balls in $\R^d$ centered at the
origin and set
%
\begin{align*}
  \Delta_p(\eta)
  &\vcentcolon=
  \Delta_p(\cB_p, \eta)
  = \sup_{t \geq 0}
  \P( t < \|T\|_p \leq t + \eta ).
\end{align*}
%

\begin{proposition}[Distributional approximation of
  martingale $\ell^p$-norms]
  \label{pro:application_lp}

  Assume the setup of Corollary~\ref{cor:sa_martingale},
  with $\Sigma$ non-random. Then for $T \sim \cN(0, \Sigma)$,
  %
  \begin{equation}%
    \label{eq:application_lp}
    \sup_{t \geq 0}
    \big| \P( \|S\|_p \leq t )
    - \P\left( \|T\|_p \leq t \right) \big|
    \leq \inf_{\eta>0}
    \big\{\Gamma_p(\eta) + \Delta_p(\eta) \big\}.
  \end{equation}
  %
\end{proposition}

\begin{proof}[Proposition~\ref{pro:application_lp}]

  Applying Proposition~\ref{pro:clt}
  with $\cA=\cB_p$ gives
  %
  \begin{align*}
    \sup_{t \geq 0}
    \big| \P( \|S\|_p \leq t )
    - \P\left( \|T\|_p \leq t \right) \big|
    &= \sup_{A\in \cB_p}
    \big|\P(S\in A) -\P(T\in A)\big| \\
    &\leq
    \inf_{\eta>0}
    \big\{\Gamma_p(\eta) + \Delta_p(\cB_p, \eta) \big\}
    \leq
    \inf_{\eta>0}
    \big\{\Gamma_p(\eta) + \Delta_p(\eta) \big\}.
  \end{align*}
  %
\end{proof}

The right-hand side of
\eqref{eq:application_lp} can be controlled in various ways.
%
In the case of $p=\infty$,
note that $\ell^\infty$-balls are rectangles so
$\cB_\infty\subseteq \cR$
and \eqref{eq:rect_anticonc} applies, giving
$\Delta_\infty(\eta) \leq \eta (\sqrt{2\log d} + 2) / \sigma_{\min}$
whenever $\min_j \Sigma_{j j} \geq \sigma_{\min}^2$.
Alternatively, \citet[Theorem~1]{giessing2023anti} provides
$\Delta_\infty(\eta) \lesssim \eta / \sqrt{\Var[\|T\|_\infty] + \eta^2}$.
In fact, by H{\"o}lder duality of $\ell^p$-norms, we can write
$\|T\|_p = \sup_{\|u\|_q \leq 1} u^\T T$ where
$1/p + 1/q = 1$.
Then, applying the Gaussian process anti-concentration result of
\citet[Theorem~2]{giessing2023anti} yields the more general
$\Delta_p(\eta) \lesssim \eta / \sqrt{\Var[\|T\|_p] + \eta^2}$.
Thus, the problem can be reduced to that of bounding
$\Var\left[\|T\|_p\right]$, with techniques for doing so
discussed, for example, in
\citet[Section~4]{giessing2023anti}.
Note that alongside the $\ell^p$-norms,
other functionals can be analyzed in this manner,
including the maximum statistic
and other order statistics
\citep{kozbur2021dimension,giessing2023anti}.

To conduct inference in this situation, we need to feasibly
approximate the quantiles of $\|T\|_p$.
To that end, take a significance level $\tau\in(0,1)$ and define
%
\begin{equation*}
  \hat q_p(\tau)
  = \inf \big\{t \in \R:
  \P(\|\hat T\|_p \leq t \mid \bX) \geq \tau \}
  \quad\text{where}\quad
  \hat T \mid \bX \sim \cN(0, \hat\Sigma),
\end{equation*}
%
with $\hat\Sigma$ any $\bX$-measurable positive semi-definite
estimator of $\Sigma$.
Note that for the canonical estimator $\hat\Sigma = \sum_{i=1}^n X_i X_i^\T$
we can write $\hat T =\sum_{i=1}^n X_i Z_i$ with
$Z_1,\dots,Z_n$ i.i.d.\ standard Gaussian independent of $\bX$,
yielding the Gaussian multiplier bootstrap.
Now assuming
the law of $\|\hat T\|_p \mid \bX$ has no atoms,
we can apply Proposition~\ref{pro:bootstrap}
to see
%
\begin{align*}
  &\sup_{\tau\in(0,1)}
  \big|\P\left(\|S\|_p \leq \hat q_p(\tau)\right) - \tau \big|
  \leq
  \E\left[
    \sup_{t \geq 0}
    \big|
    \P(\|S\|_p \leq t)
    - \P(\|\hat T\|_p \leq t \mid \bX)
    \big|
  \right] \\
  &\qquad\leq
  \inf_{\eta>0}
  \left\{ \Gamma_p(\eta)
    + 2 \Delta_p(\eta)
    + 2d\, \E\left[
      \exp\left(\frac{-\eta^2}
      {2d^{2/p}\big\|\hat \Sigma^{1/2} - \Sigma^{1/2}\big\|_2^2}\right)
    \right]
  \right\}
\end{align*}
%
and hence the bootstrap is valid whenever
$\|\hat \Sigma^{1/2} - \Sigma^{1/2}\big\|_2^2$ is sufficiently small. See the
preceding discussion regarding methods for bounding this object.

\begin{remark}[One-dimensional distributional approximations]
  In our application to distributional approximation of $\ell^p$-norms,
  the object of interest $\|S\|_p$ is a
  one-dimensional functional of the high-dimensional martingale;
  contrast this with the more general Proposition~\ref{pro:clt} which
  directly considers the $d$-dimensional random vector $S$.
  As such, our coupling-based approach may be improved in certain settings
  by applying a more carefully tailored smoothing argument.
  For example, \citet{belloni2018high}
  employ a ``log sum exponential'' bound
  \citep[see also][]{chernozhukov2013gaussian}
  for the maximum statistic
  $\max_{1 \leq j \leq d} S_j$
  along with a coupling due to \citet{chernozhukov2014gaussian} to attain
  an improved dependence on the dimension.
  Naturally their approach does not permit the formulation of
  high-dimensional central limit theorems over arbitrary classes of
  Borel sets as in our Proposition~\ref{pro:clt}.
\end{remark}
