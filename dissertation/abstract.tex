%! TeX root = phd_dissertation.tex

% 350 words max

Nonparametric methods are central to modern statistics, enabling data analysis
with minimal assumptions in a wide range of scenarios. While contemporary
procedures such as random forests and kernel methods are popular due to their
performance and flexibility, their statistical properties are often less well
understood. The availability of sound inferential techniques is vital in the
sciences, allowing researchers to quantify uncertainty in their models. We
develop methodology for robust and practical statistical estimation and
inference in some modern nonparametric settings involving complex estimators
and nontraditional data.

We begin in the regression setting by studying the Mondrian random forest, a
variant in which the partitions are drawn from a Mondrian process. We present a
comprehensive analysis of the statistical properties of Mondrian random
forests, including a central limit theorem for the estimated regression
function and a characterization of the bias. We show how to conduct feasible
and valid nonparametric inference by constructing confidence intervals, and
further provide a debiasing procedure that enables minimax-optimal estimation
rates for smooth function classes in arbitrary dimension.

Next, we turn our attention to nonparametric kernel density estimation with
dependent dyadic network data. We present results for minimax-optimal
estimation, including a novel lower bound for the dyadic uniform convergence
rate, and develop methodology for uniform inference via confidence bands and
counterfactual analysis. Our methods are based on strong approximations and are
designed to be adaptive to potential dyadic degeneracy. We give empirical
results with simulated and real-world economic trade data.

Finally, we develop some new probabilistic results with applications to
nonparametric statistics. Coupling has become a popular approach for
distributional analysis in recent years, and Yurinskii's method stands out for
its wide applicability and explicit formulation. We present a generalization of
Yurinskii's coupling, treating approximate martingale data under weaker
conditions than previously imposed. We allow for Gaussian mixture coupling
distributions, and a third-order method permits faster rates in certain
situations. We showcase our results with applications to factor models and
martingale empirical processes, as well as nonparametric partitioning-based and
local polynomial regression procedures.
